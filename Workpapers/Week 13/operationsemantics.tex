\subsection{Operation Semantics}

\begin{figure}[h]
\begin{center}
\begin{align*}
	&first(M_1,M_2)=M_1\\
	&second(M_1,M_2)=M_2\\
	third(M_1,M_2,M_3)=&second(M_1,second(M_2,M_3))=M_3
\end{align*}
\end{center}
\caption{A tuple operation, where we can get a value from a tuple, by only using the two functions \textit{first} and \textit{second}, and recursively using \textit{second} to make all other tuple variants.}
\end{figure}

\begin{figure}[h]
\begin{center}
\begin{align*}
	f(1) =& first(M_1, M_2)\\
	f(n+1) =& second(M_1, f(n))
\end{align*}
\end{center}
\caption{The recursive function for getting values out of tuples. Where the number of values in the tuples is equal or greater than 2}
\end{figure}

\FloatBarrier