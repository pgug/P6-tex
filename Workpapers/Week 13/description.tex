\subsection{Description}
Our description of the syntax of the rho-calculus.

\subsubsection{Nil}
Nil is a process that does not do anything, and terminates the process.

\subsubsection{Input}
The input process receives \textit{y} on channel \textit{x} and runs afterwards the process \textit{P}.

\subsubsection{Lift}
The lift process sends \textit{M} on channel \textit{x}, and terminates afterwards.

\subsubsection{Drop}
The drop process drops \textit{x} so \textit{x} becomes a process. If \textit{x} already is a quoted process it becomes the process it was before.

\subsubsection{Parallel}
The parallel process runs process \textit{P} and process \textit{Q} in parallel.

\subsubsection{Quote}
The quote takes a process \textit{P} and it becomes a name. If \textit{P} already is a dropped name, it becomes the name it was before.

\subsubsection{Terms}
\textbf{Numbers} 
Numbers is a term which can be used for saving numbers.
\\\\
\textbf{Strings}
Strings is a term which can be used for saving a list of characters.
\\\\
\textbf{Tuples}
Tuples takes a number of terms bigger than or equal to two, and combines it into one term.
\\\\
\textbf{Operations}
Operations takes a term and returns a term. Both the input and output term can be a tuple and that way the operation can takes multiple input and output.
