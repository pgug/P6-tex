\subsection{Blockchain Code}

\subsubsection{Block}
A block in the block chain can be made of a tuple of $Terms$, like this.
\begin{align*}
    Block = (index, previousHash, timestamp, data, hash)
\end{align*}

\subsubsection{Generate New Block}
When a new block is made, the information for the new block is generated and saved like this. We have used a syntax of getting information out of tuples with a dot, but we have not yet formed the semantic rules for that. We have not made a syntax structure or semantic for the return statement, so right now the last process in chain of parallel process, should act as a return statement.
\begin{align*}
    generateNewBlock(data, previousBlock)\eqdef \\
    index[inc(previousBlock.Index)]\para\\
    previousHash[previousBlock.hash] \para \\
    timestamp[time.now] \para \\
    hash[hash(data)] \para \\
    (index(y).<y>, previousHash(x).<x>,\\ timeStamp(a).<a>, data, hash(b).<b>)
\end{align*}

\begin{align*}
generateNewBlock(data, previousBlock)\eqdef \\(index, previousHash, timeStamp, data, hash)
\end{align*}


\subsubsection{Hash Block}
If we assume that we have a standard library, with the most standard operation and cryptographic operation,to our disposal we may assume that the SHA256 operation should be one of them.
\begin{align*}
    hashBlock(block)\eqdef\\
    SHA256(block)\\
\end{align*}

\subsubsection{Add Block}
The addblock checks if the new block is valid, and if it is, then it is pushing it to the chain.
\begin{align*}
    addBlock(newBlock, previousBlock) \eqdef\\
    isValidNewBlock(newBlock, previousBlock)[\nil]\para \\
    \inp{true}{y}blockchain.push(newBlock)
\end{align*}

\subsubsection{Is Valid New Block}
is validnewblock is checking if the new block is valid, and returns true, but if it is not valid it gets stuck, and we need a way to check if things is not true.
\begin{align*}
    isValidNewBlock(newBlock, previousBlock) \eqdef\\
    inc(previousBlock.index)[\nil] \para \\
    \inp{newBlock.index}{y}trueIndex[\nil] \para \\
    previousBlock.hash[\nil] \para \\
    \inp{newBlock.previousHash}{y}trueHash[\nil] \para \\
    hashBlock(newBlock)[\nil] \para \\
    \inp{new.Block.hash}{y}trueBlock[\nil] \para \\
    \inp{trueIndex}{x}\inp{trueHash}{y}\inp{trueBlock}{z}totalTrue[true] \para \\
    \inp{totalTrue}{a}\drop{a}
\end{align*}

%\subsubsection{Is Chain Valid}

%\begin{align*}
%isChainVlaid(blockchainToValidate) \Rightarrow\\
%blockchainToValidate[0][\nil] \para \\
%\inp{genesisBlock}{y}
%\end{align*}


