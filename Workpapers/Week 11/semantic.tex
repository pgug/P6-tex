
\subsection{Structural Congruence}
\begin{align*}
P\para 0 \equiv P \equiv 0\para P\\
P\para Q \equiv Q\para P\\
P\para (Q\para R)\equiv (P\para Q)\para R
\end{align*}
The structural congruence for parallel process.

\begin{table}[h]
\begin{center}
\begin{tabular}[c]{ll}
  \runa{Equivalence} & \infrule{P\equiv P'\quad P'\ra Q'\quad Q'\equiv Q}{P\ra Q}
\end{tabular}
\end{center}
\caption{Structural Congruence for Equivalence}
\label{tab:equi}
\end{table}
\noindent
The structural equivalence for $P$ and $Q$, as seen in \tabref{tab:equi}
\FloatBarrier


\subsection{Reduction Rules}

\begin{table}[!h]
\begin{center}
\begin{tabular}[c]{ll}
  \runa{Communication} & \infrule{x_1\equiv _N x_2 }{x_1[y]\para \inp{x_2}{z}P\ra P\{y/ z\}} 
\end{tabular}
\end{center}
\caption{Reduction rule for Communication}
\label{tab:com}
\end{table}
The communication rule for input and output is seen in \tabref{tab:com}.


\begin{table}[!h]
\begin{center}
\begin{tabular}[c]{ll}
  \runa{Drop and Quote} & \infrule{}{\quot{\drop{x}}\equiv _N x}
\end{tabular}
\end{center}
\caption{Reduction rule for Drop and Quote}
\label{tab:dropquot}
\end{table}
The drop and quote rules are applied to $x$ in \tabref{tab:dropquot}. Where if $x$ if first dropped, and then quoted then it should be name equivalent with $x$.

\begin{table}[!h]
\begin{center}
\begin{tabular}[c]{ll}
  \runa{Structural equivalence} & \infrule{P\equiv Q}{\quot{P} \equiv _N \quot{Q}}
\end{tabular}
\end{center}
\caption{Reduction rule for Structural equivalence}
\label{tab:strucequiv}
\end{table}
The structural equivalence of $P$ and $Q$, where if $P$ and $Q$ are congruent with each other, the quoted version of $P$ and $Q$ are name equivalent with each other, see \tabref{tab:strucequiv}.

\begin{table}[!h]
\begin{center}
\begin{tabular}[c]{ll}
  \runa{Parallel} & \infrule{P\ra P'}{P\para Q\ra P'\para Q}
\end{tabular}
\end{center}
\caption{Reduction rule for Parallel}
\label{tab:para}
\end{table}
The reduction rule for $P$ parallel with $Q$ as seen in \tabref{tab:para}.
\FloatBarrier
