\subsection{Description}
Our description of the syntax of the rho-calculus.

\subsubsection{Proces}
\textbf{Nil}
Nil is a process that does not do anything, and terminates the process.
\\
\textbf{Input}
The input process receives \textit{y} on channel \textit{x} and runs afterwards the process \textit{P}.
\\
\textbf{Lift}
The lift process sends \textit{M} on channel \textit{x}, and terminates afterwards.
\\
\textbf{Drop}
The drop process drops \textit{x} so \textit{x} becomes a process.
\\
\textbf{Parallel}
The parallel process runs process \textit{P} and process \textit{Q} in parallel.

\subsubsection{Names}
\textbf{Quote}
The quote takes a process \textit{P} and it becomes a name.

\subsubsection{Terms}
\textbf{Numbers} 
Numbers is a term for numbers.
\\
\textbf{Strings}
Strings is a term for text.
\\
\textbf{Tuples}
Tuples takes a number of terms bigger than or equal to two, and combines it into one term.
\\
\textbf{Operations}
Operations takes a term and returns a term. Both the input and output term can be a tuple and that way the operation can takes multiple input and output.
