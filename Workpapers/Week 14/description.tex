\subsection{Description}
Our description of the syntax of the rho-calculus.

\subsubsection{Process}
\begin{description}
\item[Nil] Nil is a process that does not do anything.
\item[Input] The input process receives \textit{y} on channel \textit{x} and runs afterwards the process \textit{P}.
\item[Lift]The lift process sends \textit{M} on channel \textit{x}, and terminates afterwards.
\item[Drop] The drop process drops \textit{x} so \textit{x} becomes a process.
\item[Parallel] The parallel process runs process \textit{P} and process \textit{Q} in parallel.
\end{description}


\subsubsection{Names}
\begin{description}
\item[Quote] The quote takes a process \textit{P} and it becomes a name.
\end{description}


\subsubsection{Terms}
\begin{description}
\item[Numbers] Numbers is a term for numbers.
\item[Strings] Strings is a term for text.
\item[Tuples] Tuples is a set of \textit{terms}, where the numbers of \textit{terms} is bigger than or equal to two.
\item[Operations] Operations is a way we can describe function or procedures.
\end{description}

%\subsubsection{Boolean Arithmetic}

