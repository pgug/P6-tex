\subsubsection{Tuple operation}
The operation for getting terms out of a tuple could look like the projection in \figref{tupleop}, where \textit{k} determines the index starting from 1.
\begin{figure}[h]
    \begin{align*}
        \pi^k((M_1,..., M_n))=M_k \text{ If } 1\leq k \leq n
    \end{align*}
    \caption{The operation for projecting a tuple. Where \textit{M} is a tuple, and \textit{k} is the index and bigger than  or equal to 1}
\label{tupleop}
\end{figure}

In our implementation we use tuples to represent a group of attributes. To project an attribute from a tuple, we use the name of the attribute instead of a number. As an example, projecting the hash of a block can be expressed as $\pi^{hash}(block)$.
\FloatBarrier