\subsubsection{Tuple operation}
The operation for getting terms out of a tuple could look like this recursive operation, where \textit{n} determines the index starting from 1.
\begin{figure}[h]
    \begin{center}
        \begin{align*}
            first&(M_1, M_2) = M_1\\
            second&(M_1, M_2) = M_2\\
            \\
            f(1) \eqdef & first(M_1, M_2)\\
            f(n) \eqdef & second(M_1, f(n-1))\\
            \\
            index(M, n) \eqdef &[n = 1]first(M)\para \\
            &[n > 1]index(second(M), n-1)
        \end{align*}
    \end{center}
    \caption{The operation for indexing a tuple, where \textit{M} is a tuple, and \textit{n} is a number bigger than  or equal to 1}
\end{figure}
\FloatBarrier
