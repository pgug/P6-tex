%Doc settings
\documentclass[12pt,a4paper]{article}
\usepackage[utf8]{inputenc}
\usepackage[english]{babel}
\usepackage{lmodern} %vector font Latin Modern
\usepackage[T1]{fontenc} %font encoding
\usepackage{amsmath}
\usepackage{amssymb}
\usepackage{stmaryrd}
\usepackage{placeins}

% Bibliography
\usepackage{natbib}
\setcitestyle{numbers,square}
\bibliographystyle{unsrtnat}
% Proceskonstruktioner

\newcommand{\quot}[1]{\ensuremath{\ulcorner  #1  \urcorner}}
\newcommand{\drop}[1]{\ensuremath{\urcorner  #1  \ulcorner}}
\newcommand{\lift}[2]{\ensuremath{#1 \langle \! |  #2  | \! \rangle}}
\newcommand{\nil}{\ensuremath{\mathbf{0}}}


\newcommand{\inp}[2]{\ensuremath{#1(#2).}}
\newcommand{\para}{\ensuremath{\mid}}

% Inferensregler og regelnavne

\newcommand{\infrule}[2]  {\parbox{4.5cm}{$$ \frac{#1}{#2}\hspace{.5cm}$$}}
\newcommand{\runa}[1]{\textsc{(#1})}

% Pile

\newcommand{\ra}{\rightarrow}

% Henvisninger

\newcommand{\tabref}[1]{Table \ref{#1}}
%\renewcommand{\figref}[1]{Figure \ref{#1}}

\newcommand{\redchain}[3]{
\begin{table}[!h]
\begin{center}
\begin{tabular}{cll}
$$#1$$
\end{tabular}
\end{center}
\caption{#2}
\label{#3}
\end{table}
}

\newcommand{\dsb}[1]{\llbracket #1 \rrbracket}

\newcommand{\decode}[1]{\ensuremath{\langle#1\rangle}}

\newcommand{\eqdef}{\overset{\underset{\mathrm{def}}{}}{=}}

\newcommand{\If}[3]{\ensuremath{\text{ if }#1\text{ then }#2\text{ else }#3}}

\newcommand{\fig}[3]{
\begin{figure}[h]
\begin{align*}
#1
\end{align*}
\caption{#2}
\label{#3}
\end{figure}
}

\newcommand{\op}[1]{\text{\sffamily #1\normalfont}}
\newcommand{\mop}[1]{\text{\sffamily \textit{#1}\normalfont}}
\newcommand{\level}[1]{\ensuremath{\mathcal{L}(#1)}}
\newcommand{\valid}[1]{\ensuremath{\mathcal{V}(#1)}}
\setlength{\parindent}{0pt}
\author{Group D608F17}
\title{Workpaper week 14}


\begin{document}
\maketitle
\section{Rho-Calculus}
\section{Rho-Calculus}\label{ch:rho-calculus}
We are using rho-calculus because it is a small and simple calculus which should be fast to learn.
It is developed and described by L. G. Meredith and M. Radestock in \citep{Meredith2005}.
We will use rho-calculus to model a system, and later a security type rule will be applied. We could maybe make the calculus even smaller but that would make it more difficult to express and model a system in it, we would actually want to expand it a little to make it easier to model a system in it.
Because rho-calculus is a small calculus, there will not be many reduction rules to look through, and the type rules for the reduction rules will then be few compared a bigger calculus.
The rho-calculus has a syntax for parallelism which will be useful for modelling client/server systems, where security can be an issue, and security can be a major part of the specification. 
%Our syntax is similar to the one described in \citep{Meredith2005}, however our lift includes a term $M$ that will be explained in section \ref{sec:addsyntax}.
The syntax here is as the syntax described in \citep{Meredith2005}, we would like to add some syntax rules on top on this, to make it easier to implement systems in the calculus.
\begin{align*}
    P  ::= \; &  \nil & \text{nil} \\
      & \mid \inp{x}{y}P & \text{input} \\
      & \mid \lift{x}{P} & \text{lift} \\
      & \mid \drop{x} & \text{drop} \\
      & \mid P \para Q & \text{parallel} \\[3mm]
    x,y ::= \; & \quot{P} & \text{quote}\\
\end{align*}

\subsection{Description}
Our description of the syntax of the rho-calculus.

\subsubsection{Nil}
Nil is a process that does not do anything, and terminates the process.

\subsubsection{Input}
The input process receives \textit{y} on channel \textit{x} and runs afterwards the process \textit{P}.

\subsubsection{Lift}
The lift process sends \textit{M} on channel \textit{x}, and terminates afterwards.

\subsubsection{Drop}
The drop process drops \textit{x} so \textit{x} becomes a process. If \textit{x} already is a quoted process it becomes the process it was before.

\subsubsection{Parallel}
The parallel process runs process \textit{P} and process \textit{Q} in parallel.

\subsubsection{Quote}
The quote takes a process \textit{P} and it becomes a name. If \textit{P} already is a dropped name, it becomes the name it was before.

\subsubsection{Terms}
\textbf{Numbers} 
Numbers is a term which can be used for saving numbers.
\\\\
\textbf{Strings}
Strings is a term which can be used for saving a list of characters.
\\\\
\textbf{Tuples}
Tuples takes a number of terms bigger than or equal to two, and combines it into one term.
\\\\
\textbf{Operations}
Operations takes a term and returns a term. Both the input and output term can be a tuple and that way the operation can takes multiple input and output.


\subsection{Additional Syntax} \label{sec:addsyntax}
Although it is possible to implement a blockchain protocol in the rho-calculus, it would require convoluted structures to express even simple concepts.
We choose to extend the rho-calculus to allow easier modelling of those concepts.
We can do this with out making the rho-calculus stronger, because in the simple pi-calculus you can code a calculus with terms\citep{Baldamus2005}, and the simple pi-calculus can be coded in the rho-calculus\citep{Meredith2005}.\\
\\
We add an additional syntax rule for terms, where $M$ ranges over terms, $n$ ranges over numbers, $s$ ranges over strings and $f$ ranges over operations. We have also added a possibility of lifting a term, instead of a process.

In addition to terms and processes we need to use conditional processes. This requires boolean expressions, denoted by \ensuremath{\phi}.
\begin{align*}
P  ::= \; &  \nil & \text{nil} \\
      & \mid \inp{x}{y}P & \text{input} \\
	  & \mid \lift{x}{M}&\text{lift}\\
      & \mid \drop{x} & \text{drop} \\
      & \mid P \para Q & \text{parallel} \\
      & \mid [\phi] P\\[3mm]
    x,y ::= \; & \quot{P} & \text{quote}\\[3mm]
M::=\; & n &\text{number}\\
 	  &\mid s &\text{string}\\
 	  &\mid (M_1,...,M_k)\quad\quad\quad \text{for all k $\geq$ 2} &\text{tuple}\\
 	  &\mid fM &\text{operation}\\
 	  &\mid x &\text{name}\\[3mm]
\phi ::=& \mid M_1\gamma M_2 \mid \phi\land\phi \mid \phi\lor\phi \mid \neg\phi \mid \top \mid \bot\\[3mm]
\gamma ::=& \mid = \mid \neq \mid < \mid > \mid \leq \mid \geq
\end{align*}

To ease the writing of conditional branching, we allow an if-then-else syntax.
\begin{align*}
	\If{\phi}{P}{Q} \eqdef [\phi].P \mid [\neg \phi].Q
\end{align*}



\subsection{Description}
Our description of the syntax of the rho-calculus.

\subsubsection{Nil}
Nil is a process that does not do anything, and terminates the process.

\subsubsection{Input}
The input process receives \textit{y} on channel \textit{x} and runs afterwards the process \textit{P}.

\subsubsection{Lift}
The lift process sends \textit{M} on channel \textit{x}, and terminates afterwards.

\subsubsection{Drop}
The drop process drops \textit{x} so \textit{x} becomes a process. If \textit{x} already is a quoted process it becomes the process it was before.

\subsubsection{Parallel}
The parallel process runs process \textit{P} and process \textit{Q} in parallel.

\subsubsection{Quote}
The quote takes a process \textit{P} and it becomes a name. If \textit{P} already is a dropped name, it becomes the name it was before.

\subsubsection{Terms}
\textbf{Numbers} 
Numbers is a term which can be used for saving numbers.
\\\\
\textbf{Strings}
Strings is a term which can be used for saving a list of characters.
\\\\
\textbf{Tuples}
Tuples takes a number of terms bigger than or equal to two, and combines it into one term.
\\\\
\textbf{Operations}
Operations takes a term and returns a term. Both the input and output term can be a tuple and that way the operation can takes multiple input and output.

\subsection{Equivalence Relations}
The structural congruence of processes is denoted by the relation $\equiv$.

\begin{align*}
	P\para 0 \equiv\ &P \equiv 0\para P\\
    P\para Q &\equiv Q\para P\\
    P\para (Q\para R)&\equiv (P\para Q)\para R
\end{align*}


\FloatBarrier

The equivalence of names is denoted by the relation $\equiv _N$

\begin{align}
	& \infrule{}{\quot{\drop{x}}\equiv _N x} \tag{Drop and Quote}
\end{align}

\noindent
The rule states that if $x$ is dropped, and then quoted, then it should stay name equivalent with $x$.

\begin{align}
	& \infrule{P\equiv Q}{\quot{P} \equiv _N \quot{Q}} \tag{Structural equivalence}
\end{align}

\noindent
The rule states, that if $P$ and $Q$ are structural congruent with each other, then the quoted $P$ and $Q$ are name equivalent with each other.

\FloatBarrier

\subsection{Reduction Rules}
We make use of the following reduction rules.

\subsubsection{Base rules}
These are similar to the reduction rules of the rho-calculus as described in \citep{Meredith2005}. The only difference is the communication rule reflecting the syntax changes.

\begin{align}
	\tag{Comm} \infrule{x_0 \equiv_N x_1}{\lift{x_0}{M}\para\inp{x_1}{y}P\ra P\{M/y\}}&\\
	\tag{Parallel} \infrule{P\ra P'}{P\para Q\ra P'\para Q}&\\
	\tag{Equivalence} \infrule{P\equiv P'\quad P'\ra Q'\quad Q'\equiv Q}{P\ra Q}&
\end{align}

\FloatBarrier

\subsubsection{Rules for conditional processes}
The reduction rule of a given conditional process is decided by its boolean expression. These are evaluated using common boolean arithmetics.

\begin{align}
	& \infrule{\phi \ra \top}{[\phi]P\ra P} \tag{True}\\
	& \infrule{\phi \ra \bot}{[\phi]P\ra \nil} \tag{False}
\end{align}
	

\FloatBarrier
We also make use of replication, denoted by $!P$, to express inexhaustible processes.
The replication $!P$ proceeds as an arbitrary number of copies of $P$ in parallel. This is shown by the reduction example in \figref{fig:reductionexample}. The replication was developed by L. G. Meredith and M. Radestock\citep{Meredith2005}.
%\begin{figure}[h]
%    \begin{center}
%        \begin{tabular}[c]{cll}
%            & !P & \runa{Initial} \\
%            $\equiv$ & \lift{x}{\inp{x}{y}(\lift{x}{\drop{y}}\para\drop{y})\para P}\para\inp{x}{y}(\lift{x}{\drop{y}}\para \drop{y}) & \runa{Substitution} \\

%            $\ra$ & \lift{x}{\drop{\quot{\inp{x}{y}(\lift{x}{\drop{y}}\para\drop{y})\para P}}}\para\drop{\quot{\inp{x}{y}(\lift{x}{\drop{y}}\para\drop{y})\para P}} & \runa{Communication} \\

%            $\equiv$ & \lift{x}{\inp{x}{y}(\lift{x}{\drop{y}}\para\drop{y})\para P}\para\inp{x}{y}(\lift{x}{\drop{y}}\para \drop{y})\para P & \runa{DropQuote} \\

%            $\equiv$ & !P\para P & \runa{Substitution}
%        \end{tabular}
%    \end{center}
%    \caption{Reduction example of replication}
%    \label{fig:reductionexample}
%\end{figure}


\begin{figure}[h]
    \begin{align}
        &!P \tag{Initial} \\
        &\equiv \lift{x}{\inp{x}{y}(\lift{x}{\drop{y}}\para\drop{y})\para P}\para\inp{x}{y}(\lift{x}{\drop{y}}\para \drop{y}) \tag{Substitution} \\
        &\ra \lift{x}{\drop{\quot{\inp{x}{y}(\lift{x}{\drop{y}}\para\drop{y})\para P}}}\para\drop{\quot{\inp{x}{y}(\lift{x}{\drop{y}}\para\drop{y})\para P}} \tag{Comm}\\
        &\equiv \lift{x}{\inp{x}{y}(\lift{x}{\drop{y}}\para\drop{y})\para P}\para\inp{x}{y}(\lift{x}{\drop{y}}\para \drop{y})\para P \tag{DropQuote} \\
        &\equiv\ 
        !P\para P \tag{Substitution}
    \end{align}
    \caption{Reduction example of replication}
    \label{fig:reductionexample}
\end{figure}

\FloatBarrier

%\begin{itemize}
%    \item encrypt --- enc
%    \item decrypt --- dec
%    \item equal   --- eq
%    \item !equal  --- !eq
%    \item greater --- >
%    \item greatereq --- >=
%    \item lesser  --- <
%    \item lessereq --- <=
%    \item hashing --- hash
%\end{itemize}

\begin{align}
    &enc(M_1, M_2) = M_2 \tag{Encrypt}\\
    &dec(enc(M_1, M_2),M_2) = M_1 \tag{Decrypt}\\
    &hash(M_1) = M_2 \tag{Hashing}\\
    &[x_1 = x_2]P \ra P \tag{Equal}\\
    &[x_1 \neq x_2]P \ra P \tag{Not Equal}\\
    &[x_1 > x_2]P \ra P \tag{Greater}\\
    &[x_1 >= x_2]P \ra P \tag{Greater than}\\
    &[x_1 < x_2]P \ra P \tag{Lesser}\\
    &[x_1 <= x_2]P \ra P \tag{Lesser than}
\end{align}


\subsection{Reduction Example}

\begin{table}[!h]
\begin{center}
\begin{tabular}[c]{cll}
     & !P & \runa{Initial} \\
    
     $\equiv$ & \lift{x}{\inp{x}{y}(\lift{x}{\drop{y}}\para\drop{y})\para P}\para\inp{x}{y}(\lift{x}{\drop{y}}\para \drop{y}) & \runa{Substitution} \\
    
     $\ra$ & \lift{x}{\drop{\quot{\inp{x}{y}(\lift{x}{\drop{y}}\para\drop{y})\para P}}}\para\drop{\quot{\inp{x}{y}(\lift{x}{\drop{y}}\para\drop{y})\para P}} & \runa{Communication} \\
    
     $\equiv$ & \lift{x}{\inp{x}{y}(\lift{x}{\drop{y}}\para\drop{y})\para P}\para\inp{x}{y}(\lift{x}{\drop{y}}\para \drop{y})\para P & \runa{DropQuote} \\
    
     $\equiv$ & !P\para P & \runa{Substitution}
\end{tabular}
\end{center}
\caption{Reduction example of replication}
\label{tab:reductionexample}
\end{table}

\subsection{Evaluating Terms}
We have found that some operations are useful to have defined, when coding a blockchain protocol in rho-calculus. These will be described in this section.
\subsubsection{Tuple operation}
The operation for getting Terms out of a tuple could look like this recursive operation, where \textit{n} determines the index starting from 1.  
\begin{figure}[h]
    \begin{center}
        \begin{align*}
            first&(M_1, M_2) = M_1\\
            second&(M_1, M_2) = M_2\\
            \\
            f(1) \eqdef & first(M_1, M_2)\\
            f(n) \eqdef & second(M_1, f(n-1))\\
            \\
            index(M, n) \eqdef &[n = 1]first(M)\para \\
            &[n > 1]index(second(M), n-1)
        \end{align*}
    \end{center}
    \caption{The operation for indexing a tuple. Where \textit{M} is a tuple, and \textit{n} is a number bigger than  or equal to 1}
\end{figure}
\FloatBarrier

%\begin{itemize}
%    \item encrypt --- enc
%    \item decrypt --- dec
%    \item equal   --- eq
%    \item !equal  --- !eq
%    \item greater --- >
%    \item greatereq --- >=
%    \item lesser  --- <
%    \item lessereq --- <=
%    \item hashing --- hash
%\end{itemize}

\begin{align}
    &enc(M_1, M_2) = M_2 \tag{Encrypt}\\
    &dec(enc(M_1, M_2),M_2) = M_1 \tag{Decrypt}\\
    &hash(M_1) = M_2 \tag{Hashing}\\
    &[x_1 = x_2]P \ra P \tag{Equal}\\
    &[x_1 \neq x_2]P \ra P \tag{Not Equal}\\
    &[x_1 > x_2]P \ra P \tag{Greater}\\
    &[x_1 >= x_2]P \ra P \tag{Greater than}\\
    &[x_1 < x_2]P \ra P \tag{Lesser}\\
    &[x_1 <= x_2]P \ra P \tag{Lesser than}
\end{align}

\subsection{Type Security}

\fig{M: secret,c:secure\vdash \lift{c}{M}}{If \textit{M} is of type \textit{secret}, then \textit{c} should be of type \textit{secure}}{fig:sec}

\fig{\frac{\Gamma \vdash M:Secret \quad \Gamma \vdash c:Secure}{\Gamma \vdash \lift{c}{M}}}{}{}

\fig{\frac{\Gamma \vdash M:Public \quad \Gamma \vdash c:Insecure}{\Gamma \vdash \lift{c}{M}}}{}{}



\subsection{Implementation of blockchain}

\begin{align*}
    User &= !mineBlock \lift{}{data} |!addPeer\lift{}{address}\\
    MineBlock &= !mineBlock(data).blockchain(chain).(blockchain\\
        &\lift{}{addToChain(newBlock(data,getLatest(chain)),chain)}\\
        &|\overline{blockUpdate})\lift{}{newBlock(data, getLatest(chain))})\\
    QueryLatest &=!queryLatest(null).blockchain(chain).(blockchain\\
        &\lift{}{chain}|\overline{blockUpdate}\lift{}{getLatest(chain)})\\
        QueryAll &= !queryAll(null).blochchain(chain).blockchain\lift{}{chain}|\\
        &\overline{blockUpdate}\lift{}{chain})\\
    BlockUpdate &= !blockUpdate(chain).blockchain(localchain).[indexOf(\\
        &getLatest(chain))>indexOf(getLatest(localchain)),[hashOf(getLatest(\\
        &localchain))=prevHashOf(getLatest(chain)),P_0,\\
        &[lenghtOf(chain)=1,P_1,P_1]],P_3]\\
    P_0 &= blockchain\lift{}{addToChain(getLatest(chain),localchain)}|\\
        &\overline{blockUpdate}\lift{}{getLatest(chain)}\\
    P_1 &= P_3|\overline{queryAll}\lift{}{0}\\
    P_2 &= [isValid(chain) \land lenghtOf(chain) > lenghtOf(localchain),\\
        &blockchain\lift{}{chain}|\overline{blockUpdate}\lift{}{getLatest\\
        &(chain)},P_3]\\
    P_3 &= blockchain\lift{}{blockchain}
\end{align*}

\subsection{Operations}

\begin{align*}
    newBlock(data,latest)\\
    getLatest(chain)\\
    addToChain(block,chain)\\
    indexOf(block)\\
    hashOf(block)\\
    prevHashOf(block)\\
    lenghtOf(chain)\\
    isValid(chain)\\
    xOf((M_0,M_1,...M_N))=M_x
\end{align*}

\subsection{Condition rule}

\begin{align*}
    if &= [cond,True,False] = [cond].True|[\neg cond].False
\end{align*}

\subsection{Peer network}

\begin{align*}
    &(blockUpdate, queryLatest, queryAll)\\
    AddPeer &= !addPeer(peer).(peers(localpeers).peers\lift{}{(peer,localpeers)}\\
        &|queryLatestOf(peer)\lift{}{0})\\
    \overline{BlockUpdate} &= !\overline{blockUpdate}.(data).peers(localpeers).\\
        &(peers\lift{}{localpeers}|blockUpdateLoop\lift{}{(localpeers,data)})|\\
        &!blockUpdateLoop(x).(getBlockUpdate(head(getLocalPeers(x)))\\
        &\lift{}{getData(x)}|[tail(getLocalPeers(x))\not= 0,\\
        &blockUpdateLoop\lift{}{(tail(getLocalPeers(x)),getData(x))},\nil)
\end{align*}

\subsection{Client}

\begin{align*}
    Client &= MineBlock|QueryLatest|QueryAll|BlockUpdate|\overline{BlockUpdate}\\
    &|\overline{QueryLatest}|\overline{QueryAll}|peers\lift{}{0}|blockchain\lift{}{newBlockchain}
\end{align*}

%consensus for the blockchain
%secure transactions
%encrytped authentication
%digital signatures
%public key encryption
%distributed ledgers based on consensus
%smart contracts
%fault tolerant transactions processing
%zero-knowledge proofs

\subsection{Distributed fault tolerance}
Since the blockchain protocol is a distributed database, then the protocol must also contain fault tolerance regarding problems like, a node in the network dies or is attacked, a miner encounters an error during a mining process etc.
%lidt længere og mere hvorfor det her er skide vigtigt

\subsection{Immutability}
The blockchain is all about having a database that has a unmodified history, which it also uses to verify new blocks, therefor the blockchain or rather the blocks in the chain must be designed to be immutable. The blocks must never be modified after creation, which is very relevant for use cases of currency. Changing blocks would mean that users could change already made transactions and thereby use already used money which would not make sense.
%skal have en bedre start, og 1 eller 2 argumenter ekstra

\subsection{Consensus}
Consensus is the process by which the protocol, or rather the nodes in the network agrees whenever the new block is valid or not, and is allowed to be added to the chain. The popular choices of consensus protocols are proof-of-stake and proof-of-work. The proof-of-work protocol is proved to be to be valid, but the proof-of-stake is still under evaluation, and therefor relevant to analyze.
%lidt rodet men kan skal bare lige rydes op

\subsection{Authenticity}
Authenticity is very important for databases, since user rights are often used for different access rights. Zero knowledge proof is a way to verify whether a user, or in blockchains case a node, is who it claims to be. The advantage with zero knowledge proof is that no information is shared. Zero knowledge proofs works as such:\\
A prover tries to prove his claim to a verifier. The verifier then has a challenge that the prover must be able to clear to convince the verifier that the provers claim is true. To clear the challenge the prover needs some information, which the verifier wants to confirm whether the prover has, and if not the prover can clear the challenge. Now the verifier does not to actually see the information, that the prover claims to own since it is very important that the information is not leaked.  To clear the challenge the prover needs some information, which the verifier wants to confirm whether the prover has, and if not the prover can clear the challenge. Now the verifier does not to actually see the information, that the prover claims to own since it is very important that the information is not leaked. If the prover clears the challenge, the verifier will ask the challenge once more and this will continue until the verifier is convinced that the provers claim is true.
%Nok bedre med en bedre forklaring


\bibliography{../../Bibliography/Bib}
\end{document}\grid
\grid
