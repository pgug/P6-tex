%Doc settings
\documentclass[12pt,a4paper]{article}
\usepackage[utf8]{inputenc}
\usepackage[english]{babel}
\usepackage{lmodern} %vector font Latin Modern
\usepackage[T1]{fontenc} %font encoding
\usepackage{amsmath}
\usepackage{amssymb}
\usepackage{stmaryrd}

% Bibliography
\usepackage{natbib}
\setcitestyle{numbers,square}
\bibliographystyle{unsrtnat}
% Proceskonstruktioner

\newcommand{\quot}[1]{\ensuremath{\ulcorner  #1  \urcorner}}
\newcommand{\drop}[1]{\ensuremath{\urcorner  #1  \ulcorner}}
\newcommand{\lift}[2]{\ensuremath{#1 \langle \! |  #2  | \! \rangle}}
\newcommand{\nil}{\ensuremath{\mathbf{0}}}

\newcommand{\inp}[2]{\ensuremath{#1(#2).}}
\newcommand{\para}{\ensuremath{\mid}}

% Inferensregler og regelnavne

\newcommand{\infrule}[2]  {\parbox{4.5cm}{$$ \frac{#1}{#2}\hspace{.5cm}$$}}
\newcommand{\runa}[1]{\textsc{(#1})}

% Pile

\newcommand{\ra}{\rightarrow}

% Henvisninger

\newcommand{\tabref}[1]{Table \ref{#1}}


\newcommand{\redchain}[3]{
\begin{table}[!h]
\begin{center}
\begin{tabular}{cll}
$$#1$$
\end{tabular}
\end{center}
\caption{#2}
\label{#3}
\end{table}
}

\setlength{\parindent}{0pt}
\author{Group D608F17}
\title{Workpaper week 14}


\begin{document}
\maketitle
\section{Rho-Calculus}
\section{Rho-Calculus}\label{ch:rho-calculus}
We are using rho-calculus because it is a small and simple calculus which should be fast to learn.
It is developed and described by L. G. Meredith and M. Radestock in \citep{Meredith2005}.
We will use rho-calculus to model a system, and later a security type rule will be applied. We could maybe make the calculus even smaller but that would make it more difficult to express and model a system in it, we would actually want to expand it a little to make it easier to model a system in it.
Because rho-calculus is a small calculus, there will not be many reduction rules to look through, and the type rules for the reduction rules will then be few compared a bigger calculus.
The rho-calculus has a syntax for parallelism which will be useful for modeling client/server systems, where security can be an issue, and security can be a major part of the specification. 
%Our syntax is similar to the one described in \citep{Meredith2005}, however our lift includes a term $M$ that will be explained in section \ref{sec:addsyntax}.
The syntax here is as the syntax described in \citep{Meredith2005}, we would like to add some syntax rules on top on this, to make it easier to implement systems in the calculus.
\begin{align*}
    P  ::= \; &  \nil & \text{nil} \\
      & \mid \inp{x}{y}P & \text{input} \\
      & \mid \lift{x}{P} & \text{lift} \\
      & \mid \drop{x} & \text{drop} \\
      & \mid P \para Q & \text{parallel} \\[3mm]
    x,y ::= \; & \quot{P} & \text{quote}\\
\end{align*}

The syntax of the rho-calculus is described as follows.

\subsection{Processes}
\begin{description}
\item[Nil] Does nothing.
\item[Input] Receives a term lifted on channel \textit{x} and proceeds as process \textit{P} with the term replacing \textit{y}.
\item[Lift] Lifts a term \textit{M} on channel \textit{x}.
\item[Drop] Drops a name \textit{x}, so it becomes a process, and proceeds as the resulting process.
\item[Parallel] Proceeds as both process \textit{P} and process \textit{Q} in parallel.
\end{description}


\subsection{Names}
\begin{description}
\item[Quote] Quotes a process \textit{P}, so it becomes a name.
\end{description}

%\subsubsection{Boolean Arithmetic}



\subsection{Additional Syntax} \label{sec:addsyntax}
Although it is possible to implement a blockchain protocol in the rho-calculus, it would require convoluted structures to express even simple concepts.
We choose to extend the rho-calculus to allow easier modeling of those concepts.
We can do this with out making the rho-calculus stronger, because in the simple pi-calculus you can code a calculus with terms\citep{Baldamus2005}, and the simple pi-calculus can be coded in the rho-calculus\citep{Meredith2005}.\\
\\
We add an additional syntax rule for terms, where $M$ ranges over terms, $n$ ranges over numbers, $s$ ranges over strings and $f$ ranges over operations. We have also added a possibility of lifting a term, instead of a process.

In addition to terms and processes we need to use conditional processes. This requires boolean expressions, denoted by \ensuremath{\phi}.
\begin{align*}
P  ::= \; &  \nil & \text{nil} \\
      & \mid \inp{x}{y}P & \text{input} \\
	  & \mid \lift{x}{M}&\text{lift}\\
      & \mid \drop{x} & \text{drop} \\
      & \mid P \para Q & \text{parallel} \\
      & \mid [\phi] P \text{condition} \\[3mm]
    x,y ::= \; & \quot{P} & \text{quote}\\[3mm]
M::=\; & n &\text{number}\\
 	  &\mid s &\text{string}\\
 	  &\mid (M_1,...,M_k)\quad\quad\quad \text{for all k $\geq$ 2} &\text{tuple}\\
 	  &\mid fM &\text{operation}\\
 	  &\mid x &\text{name}\\[3mm]
\phi ::=& \mid M_1\gamma M_2 \mid \phi\land\phi \mid \phi\lor\phi \mid \neg\phi \mid \top \mid \bot\\[3mm]
\gamma ::=& \mid = \mid \neq \mid < \mid > \mid \leq \mid \geq
\end{align*}

To ease the writing of conditional branching, we allow an if-then-else syntax.
\begin{align*}
	\If{\phi}{P}{Q} \eqdef [\phi].P \mid [\neg \phi].Q
\end{align*}



The syntax of the rho-calculus is described as follows.

\subsection{Processes}
\begin{description}
\item[Nil] Does nothing.
\item[Input] Receives a term lifted on channel \textit{x} and proceeds as process \textit{P} with the term replacing \textit{y}.
\item[Lift] Lifts a term \textit{M} on channel \textit{x}.
\item[Drop] Drops a name \textit{x}, so it becomes a process, and proceeds as the resulting process.
\item[Parallel] Proceeds as both process \textit{P} and process \textit{Q} in parallel.
\end{description}


\subsection{Names}
\begin{description}
\item[Quote] Quotes a process \textit{P}, so it becomes a name.
\end{description}

%\subsubsection{Boolean Arithmetic}


\subsection{Equivalence Relations}
The structural congruence of processes is denoted by the relation $\equiv$.

\begin{align*}
	P\para 0 \equiv\ &P \equiv 0\para P\\
    P\para Q &\equiv Q\para P\\
    P\para (Q\para R)&\equiv (P\para Q)\para R
\end{align*}


\FloatBarrier

The equivalence of names is denoted by the relation $\equiv _N$

\begin{align}
	& \infrule{}{\quot{\drop{x}}\equiv _N x} \tag{Drop and Quote}
\end{align}

\noindent
The rule states that if $x$ is dropped, and then quoted, then it should stay name equivalent with $x$.

\begin{align}
	& \infrule{P\equiv Q}{\quot{P} \equiv _N \quot{Q}} \tag{Structural equivalence}
\end{align}

\noindent
The rule states, that if $P$ and $Q$ are structural congruent with each other, then the quoted $P$ and $Q$ are name equivalent with each other.

\FloatBarrier

\subsection{Reduction Rules}
We make use of the following reduction rules.

\subsubsection{Base rules}
These are similar to the reduction rules of the rho-calculus as described in \citep{Meredith2005}. The only difference is the communication rule reflecting the syntax changes.

\begin{align}
	\tag{Comm} \infrule{x_0 \equiv_N x_1}{\lift{x_0}{M}\para\inp{x_1}{y}P\ra P\{M/y\}}&\\
	\tag{Parallel} \infrule{P\ra P'}{P\para Q\ra P'\para Q}&\\
	\tag{Equivalence} \infrule{P\equiv P'\quad P'\ra Q'\quad Q'\equiv Q}{P\ra Q}&
\end{align}

\FloatBarrier

\subsubsection{Rules for conditional processes}
The reduction rule of a given conditional process is decided by its boolean expression. These are evaluated using common boolean arithmetics.

\begin{align}
	& \infrule{\phi \ra \top}{[\phi]P\ra P} \tag{True}\\
	& \infrule{\phi \ra \bot}{[\phi]P\ra \nil} \tag{False}
\end{align}

\begin{align*}
\infrule{\phi_1 \ra \top \quad \phi_2 \ra \top}{\phi_1 \wedge \phi2 \ra \top}\tag{And True}\\
\infrule{\phi_i \ra \top}{\phi_1 \wedge \phi_2 \ra \bot} &i\in \{1,2\}\tag{And False}\\
\infrule{\phi_i \ra \top}{\phi_1 \vee \phi2 \ra \top} &i\in \{1,2\} \tag{Or True}\\
\infrule{\phi_1 \ra \bot \quad \phi_2 \ra \bot}{\phi_1 \vee \phi2 \ra \bot} \tag{Or False}\\
\infrule{\phi \ra \bot}{\neg\phi \ra \top} \tag{Negate True}\\
\infrule{\phi \ra \top}{\neg\phi \ra \bot} \tag{Negate False}\\
\end{align*}

\begin{align*}
\infrule{}{M_1 = M_2 \ra \top}& \text{if }M_1=M_2 \tag{Equal True}\\
\infrule{}{M_1 = M_2 \ra \bot}& \text{if }M_1\neq M_2 \tag{Equal False}\\
\infrule{}{M_1 \neq M_2 \ra \top}& \text{if }M_1=M_2 \tag{NotEqual True}\\
\infrule{}{M_1 \neq M_2 \ra \bot}& \text{if }M_1\neq M_2 \tag{NotEqual False}\\
\infrule{}{M_1 < M_2 \ra \top}& \text{if }M_1<M_2 \tag{Less True}\\
\infrule{}{M_1 < M_2 \ra \bot}& \text{if }M_1\nless M_2 \tag{Less False}\\
\infrule{}{M_1 > M_2 \ra \top}& \text{if }M_1>m_2 \tag{Greater True}\\
\infrule{}{M_1 > M_2 \ra \bot}& \text{if }M_1\ngtr M_2 \tag{Greater False}\\
\infrule{}{M_1 \leq M_2 \ra \top}& \text{if }M_1\leq M_2 \tag{Less Equal True}\\
\infrule{}{M_1 \leq M_2 \ra \bot}& \text{if }M_1\nleq M_2 \tag{Less Equal False}\\
\infrule{}{M_1 \geq M_2 \ra \top}& \text{if }M_1\geq M_2 \tag{Greater Equal True}\\
\infrule{}{M_1 \geq M_2 \ra \bot}& \text{if }M_1\ngeq M_2 \tag{Greater Equal False}
\end{align*}
	

\FloatBarrier
\subsection{Replication}
Our implementation makes use of replication, denoted by $!P$, to express inexhaustible processes.
The replication $!P$ proceeds as an arbitrary number of copies of $P$ in parallel. This is shown by the reduction example in \figref{fig:reductionexample}. The replication was developed by L. G. Meredith and M. Radestock\citep{Meredith2005}.
%\begin{figure}[h]
%    \begin{center}
%        \begin{tabular}[c]{cll}
%            & !P & \runa{Initial} \\

%            $\equiv$ & \lift{x}{\inp{x}{y}(\lift{x}{\drop{y}}\para\drop{y})\para P}\para\inp{x}{y}(\lift{x}{\drop{y}}\para \drop{y}) & \runa{Substitution} \\

%            $\ra$ & \lift{x}{\drop{\quot{\inp{x}{y}(\lift{x}{\drop{y}}\para\drop{y})\para P}}}\para\drop{\quot{\inp{x}{y}(\lift{x}{\drop{y}}\para\drop{y})\para P}} & \runa{Communication} \\

%            $\equiv$ & \lift{x}{\inp{x}{y}(\lift{x}{\drop{y}}\para\drop{y})\para P}\para\inp{x}{y}(\lift{x}{\drop{y}}\para \drop{y})\para P & \runa{DropQuote} \\

%            $\equiv$ & !P\para P & \runa{Substitution}
%        \end{tabular}
%    \end{center}
%    \caption{Reduction example of replication}
%    \label{fig:reductionexample}
%\end{figure}


\begin{figure}[h]
    \begin{align}
        &!P \tag{Initial} \\
        &\equiv \lift{x}{\inp{x}{y}(\lift{x}{\drop{y}}\para\drop{y})\para P}\para\inp{x}{y}(\lift{x}{\drop{y}}\para \drop{y}) \tag{Substitution} \\
        &\ra \lift{x}{\drop{\quot{\inp{x}{y}(\lift{x}{\drop{y}}\para\drop{y})\para P}}}\para\drop{\quot{\inp{x}{y}(\lift{x}{\drop{y}}\para\drop{y})\para P}} \tag{Comm}\\
        &\equiv \lift{x}{\inp{x}{y}(\lift{x}{\drop{y}}\para\drop{y})\para P}\para\inp{x}{y}(\lift{x}{\drop{y}}\para \drop{y})\para P \tag{DropQuote} \\
        &\equiv\ 
        !P\para P \tag{Substitution}
    \end{align}
    \caption{Reduction example of replication}
    \label{fig:reductionexample}
\end{figure}

\FloatBarrier

\section{Evaluating Terms}
In order to make use of our terms we need to define how to evaluate them.
The operation terms are intended to operate on terms, so it is sufficient to define how to evaluate these operations.
Operations can also define abstract datatypes which we will make use of.
The operations defined in this section are chosen specifically for modeling a blockchain protocol and could be different given another goal.

\subsubsection{Tuple operation}
The operation for getting Terms out of a tuple could look like this recursive operation, where \textit{n} determines the index starting from 1.
\begin{figure}[h]
    \begin{align*}
        &first \quad (M_1, M_2) = M_1\\
        &second \quad (M_1, M_2) = M_2\\
        &f(1) \eqdef first(M_1, M_2)\\
        &f(n) \eqdef second(M_1, f(n-1))\\
        &index( \quad M, n) \eqdef [n = 1]first(M)\para [n > 1]index(second(M), n-1)
    \end{align*}
    \caption{The operation for indexing a tuple. Where \textit{M} is a tuple, and \textit{n} is a number bigger than  or equal to 1}
\end{figure}
\FloatBarrier


\subsection{Encryption and Decryption}
There exist no general way of retrieving the term used in an operation.
This enables us to encrypt a term by using it in an encryption operation.
We also define a decrypt operation that allows retrieval of the encrypted term.
Together with a key we can define operations for symmetric-key encryption with the rule in \figref{decryptrule}.

\begin{figure}[h]
    \begin{align*}
        &\mop{dec}(\mop{enc}(M_1, M_2),M_2) = M_1 \tag{Decrypt}
    \end{align*}
    \caption{Operations for encrypting and decrypting a message $M_1\ \mathrm{with\ key}\ M_2$.}
    \label{decryptrule}
\end{figure}
\FloatBarrier

\subsection{List}
The list operations enables the abstraction of lists.
They are described as follows:

\begin{description}
	\item[\op{head}(list)] The first element of the list
	\item[\op{tail}(list)] The list without its first element
	\item[\op{append}(x, list)] The list with element $x$ appended to its beginning.
\end{description}

These operations are defined by the rules of \figref{listoprules}.

\begin{figure}[h]
	\begin{align*}
		&\mop{head}(\mop{append}(M_1, M_2)) = M_1 \tag{Head} \\
		&\mop{tail}(\mop{append}(M_1, M_2)) = M_2 \tag{Tail}
	\end{align*}
	\caption{The rules of the list operations}
	\label{listoprules}
\end{figure}
\FloatBarrier

\section{Blockchain}

The blockchain operations enables the abstraction of blockchains.
They are described as follows:

\begin{description}
	\item[\op{newBlock}(data, previous)]
	A block that succeeds previous and contains data.
	\item[\op{getLatest}(chain)]
	The last block of the chain
	\item[\op{addToChain}(block, chain)]
	The chain with block added to it.
	\item[\op{getLength}(chain)]
	The length of the chain represented as a number.
	\item[\op{isValid}(chain)]
	The validity of the chain represented as a boolean.
\end{description}


\subsection{Reduction Example}
The replication can run \textit{P} parallel with \textit{!P} infinity many times.
%\begin{figure}[h]
%    \begin{center}
%        \begin{tabular}[c]{cll}
%            & !P & \runa{Initial} \\

%            $\equiv$ & \lift{x}{\inp{x}{y}(\lift{x}{\drop{y}}\para\drop{y})\para P}\para\inp{x}{y}(\lift{x}{\drop{y}}\para \drop{y}) & \runa{Substitution} \\

%            $\ra$ & \lift{x}{\drop{\quot{\inp{x}{y}(\lift{x}{\drop{y}}\para\drop{y})\para P}}}\para\drop{\quot{\inp{x}{y}(\lift{x}{\drop{y}}\para\drop{y})\para P}} & \runa{Communication} \\

%            $\equiv$ & \lift{x}{\inp{x}{y}(\lift{x}{\drop{y}}\para\drop{y})\para P}\para\inp{x}{y}(\lift{x}{\drop{y}}\para \drop{y})\para P & \runa{DropQuote} \\

%            $\equiv$ & !P\para P & \runa{Substitution}
%        \end{tabular}
%    \end{center}
%    \caption{Reduction example of replication}
%    \label{fig:reductionexample}
%\end{figure}


\begin{figure}[h]
    \begin{align}
        &!P \tag{Initial} \\
        &\equiv \lift{x}{\inp{x}{y}(\lift{x}{\drop{y}}\para\drop{y})\para P}\para\inp{x}{y}(\lift{x}{\drop{y}}\para \drop{y}) \tag{Substitution} \\
        &\ra \lift{x}{\drop{\quot{\inp{x}{y}(\lift{x}{\drop{y}}\para\drop{y})\para P}}}\para\drop{\quot{\inp{x}{y}(\lift{x}{\drop{y}}\para\drop{y})\para P}} \tag{Comm}\\
        &\equiv \lift{x}{\inp{x}{y}(\lift{x}{\drop{y}}\para\drop{y})\para P}\para\inp{x}{y}(\lift{x}{\drop{y}}\para \drop{y})\para P \tag{DropQuote} \\
        &\equiv !P\para P \tag{Substitution}
    \end{align}
    \caption{Reduction example of replication}
    \label{fig:reductionexample}
\end{figure}


\noindent
In \figref{fig:reductionexample} we have by reduction proved that $!P$ can run $P$ in parallel infinity many times, by applying the rule again. The Replication is an coding developed by  L. G. Meredith and M. Radestock in \citep{Meredith2005}


\subsection{Evaluating Terms}
We have found that some operations are useful to have defined, when coding a blockchain protocol in rho-calculus. These will be described in this section.
\subsubsection{Tuple operation}
The operation for getting Terms out of a tuple could look like this recursive operation, where \textit{n} determines the index starting from 1.
\begin{figure}[h]
    \begin{align*}
        &first \quad (M_1, M_2) = M_1\\
        &second \quad (M_1, M_2) = M_2\\
        &f(1) \eqdef first(M_1, M_2)\\
        &f(n) \eqdef second(M_1, f(n-1))\\
        &index( \quad M, n) \eqdef [n = 1]first(M)\para [n > 1]index(second(M), n-1)
    \end{align*}
    \caption{The operation for indexing a tuple. Where \textit{M} is a tuple, and \textit{n} is a number bigger than  or equal to 1}
\end{figure}
\FloatBarrier

\section{Evaluating Terms}
In order to make use of our terms we need to define how to evaluate them.
The operation terms are intended to operate on terms, so it is sufficient to define how to evaluate these operations.
Operations can also define abstract datatypes which we will make use of.
The operations defined in this section are chosen specifically for modeling a blockchain protocol and could be different given another goal.

\subsubsection{Tuple operation}
The operation for getting Terms out of a tuple could look like this recursive operation, where \textit{n} determines the index starting from 1.
\begin{figure}[h]
    \begin{align*}
        &first \quad (M_1, M_2) = M_1\\
        &second \quad (M_1, M_2) = M_2\\
        &f(1) \eqdef first(M_1, M_2)\\
        &f(n) \eqdef second(M_1, f(n-1))\\
        &index( \quad M, n) \eqdef [n = 1]first(M)\para [n > 1]index(second(M), n-1)
    \end{align*}
    \caption{The operation for indexing a tuple. Where \textit{M} is a tuple, and \textit{n} is a number bigger than  or equal to 1}
\end{figure}
\FloatBarrier


\subsection{Encryption and Decryption}
There exist no general way of retrieving the term used in an operation.
This enables us to encrypt a term by using it in an encryption operation.
We also define a decrypt operation that allows retrieval of the encrypted term.
Together with a key we can define operations for symmetric-key encryption with the rule in \figref{decryptrule}.

\begin{figure}[h]
    \begin{align*}
        &\mop{dec}(\mop{enc}(M_1, M_2),M_2) = M_1 \tag{Decrypt}
    \end{align*}
    \caption{Operations for encrypting and decrypting a message $M_1\ \mathrm{with\ key}\ M_2$.}
    \label{decryptrule}
\end{figure}
\FloatBarrier

\subsection{List}
The list operations enables the abstraction of lists.
They are described as follows:

\begin{description}
	\item[\op{head}(list)] The first element of the list
	\item[\op{tail}(list)] The list without its first element
	\item[\op{append}(x, list)] The list with element $x$ appended to its beginning.
\end{description}

These operations are defined by the rules of \figref{listoprules}.

\begin{figure}[h]
	\begin{align*}
		&\mop{head}(\mop{append}(M_1, M_2)) = M_1 \tag{Head} \\
		&\mop{tail}(\mop{append}(M_1, M_2)) = M_2 \tag{Tail}
	\end{align*}
	\caption{The rules of the list operations}
	\label{listoprules}
\end{figure}
\FloatBarrier

\section{Blockchain}

The blockchain operations enables the abstraction of blockchains.
They are described as follows:

\begin{description}
	\item[\op{newBlock}(data, previous)]
	A block that succeeds previous and contains data.
	\item[\op{getLatest}(chain)]
	The last block of the chain
	\item[\op{addToChain}(block, chain)]
	The chain with block added to it.
	\item[\op{getLength}(chain)]
	The length of the chain represented as a number.
	\item[\op{isValid}(chain)]
	The validity of the chain represented as a boolean.
\end{description}

\subsection{Type Security}

\fig{M: secret,c:secure\vdash \lift{c}{M}}{If \textit{M} is of type \textit{secret}, then \textit{c} should be of type \textit{secure}}{fig:sec}

\fig{\frac{\Gamma \vdash M:Secret \quad \Gamma \vdash c:Secure}{\Gamma \vdash \lift{c}{M}}}{}{}

\fig{\frac{\Gamma \vdash M:Public \quad \Gamma \vdash c:Insecure}{\Gamma \vdash \lift{c}{M}}}{}{}

\FloatBarrier

\fig{T::=l|ch^l[T]|B^l\\
B::=int|string}{The syntax of the typed security.}{}

\fig{E: names \rightharpoonup types}{}{}

\fig{E \vdash \nil}{Type rules for nil}{typenil}

\fig{\frac{E(x) \vdash ch^l[T]\quad E,y:T\vdash P}{E \vdash \inp{x}{y}P}}{Type rule for input}{typeinp}

\fig{\frac{E(x) \vdash ch^l[T] \quad E \vdash M:T}{E\vdash \lift{x}{M}}}{Type rule for lift}{typelift}

\fig{\frac{E \vdash P}{E\vdash \drop{x}}\quad where \drop{x} \equiv P}{Type rules for drop}{typedrop}

\fig{\frac{E\vdash P \quad E \vdash Q}{E \vdash P \para Q}}{Type rules for parallel}{typepara}

\fig{\frac{E \vdash x}{E\vdash \quot{P}}\quad where \quot{P} \equiv_N x}{Type rules for quot}{typequot}

\FloatBarrier

\section{Implementation of blockchain}
In this section, an implementation of a blockchain written in $\rho$-calculus will be presented. This is based on Naivechain, a naive implementation of a blockchain in javascript.

\subsection{Prerequisites}

\subsubsection{Operations}

\begin{align*}
    &newBlock(data,latest)\\
    &getLatest(chain)\\
    &addToChain(block,chain)\\
    &getLength(chain)\\
    &isValid(chain)\\
\end{align*}

\subsubsection{Condition syntax}

\begin{align*}
    if &= [cond,True,False] = [cond].True\ |\ [\neg cond].False
\end{align*}

\subsection{Overview}

\begin{align*}
	Blockchain &=\ !(User\para Client)
\end{align*}

\begin{align*}
    User &=\ !mineBlock \lift{}{data}\ |\ !addPeer\lift{}{peer}\\
    Client &= MineBlock\ |\ AddPeer\ |\ QueryLatest\ |\ QueryAll\ |\ BlockUpdate\ |\\
    &\quad \ BroadcastBlockUpdate\ |\ BroadcastQueryLatest\ |\ \ BroadcastQueryAll\ |\\
    &\quad \ peers\lift{}{emptyList}\ |\ blockchain\lift{}{newBlockchain}
\end{align*}

\subsection{Client}

\begin{align*}
    MineBlock &=\ !mineBlock(data).blockchain(chain).(blockchain\\
        &\quad \lift{}{addToChain(newBlock(data,getLatest(chain)),chain)}|\\
        &\quad \ broadcastBlockUpdate)\lift{}{newBlock(data, getLatest(chain))})\\
    QueryLatest &=\ !queryLatest(null).blockchain(chain).(blockchain\\
        &\quad \lift{}{chain}\ |\ \ broadcastBlockUpdate\lift{}{getLatest(chain)})\\
        QueryAll &=\ !queryAll(null).blockchain(chain).(blockchain\lift{}{chain}|\\
        &\quad \ broadcastBlockUpdate\lift{}{chain})\\
    BlockUpdate &=\ !blockUpdate(chain).blockchain(localchain).[indexOf(\\
        &\quad getLatest(chain))>indexOf(getLatest(localchain)),[hashOf(getLatest(\\
        &\quad localchain))=prevHashOf(getLatest(chain)),P_0,\\
        &\quad [getLength(chain)=1,P_1,P_1]],P_3]\\
    P_0 &=\ blockchain\lift{}{addToChain(getLatest(chain),localchain)}|\\
        &\quad \ broadcastBlockUpdate\lift{}{getLatest(chain)}\\
    P_1 &= P_3\ |\ broadcastQueryAll \lift{}{0}\\
    P_2 &= [isValid(chain) \land getLength(chain) > getLength(localchain),\\
        &\quad blockchain\lift{}{chain}\ |\ broadcastBlockUpdate \lift{}{getLatest(chain)},P_3]\\
    P_3 &= blockchain\lift{}{blockchain}
\end{align*}

\subsection{Peers}
The client exists in a network with an arbitrarily large number of other clients.
The subset of these clients that our client can communicate with is called its peers.
Each peer is represented as a tuple of channel names, that lets the client know where to lift messages.

\begin{align*}
    Peer &= (blockUpdate, queryLatest, queryAll)
\end{align*}

The client stores its peers as a list lifted on the channel $peers$, similar to how the blockchain is stored.
When the client receives a new peer from its user, the peer is added to its peers.

\begin{align*}
    AddPeer &=\ !addPeer(peer).(peers(localpeers).peers\lift{}{append(peer,localpeers)}|\\
        &\quad queryLatestOf(peer)\lift{}{0})\\
\end{align*}

This is done by inputting from peers into the name $localpeers$, and lifting $localpeers$ on peers again, with the new peer added.
The client also needs to query this peer for its latest block.
This is done by indexing the $queryLatest$ channel from the peer and lifting a message on it.

\subsection{Broadcasting}
In order for our client to communicate with more than one peer, we need a system that enforces each message to be lifted to all peers exactly once. This is known as broadcasting, but since there is no primitive in the rho-calculus for this, we will describe a process that enforces broadcasting.

In previous client descriptions, any message that should be broadcast, is lifted on a channel with a broadcast prefix like $broadcastBlockUpdate$.
We will describe the process, $BroadcastBlockUpdate$, that inputs messages from $broadcastBlockUpdate$ and broadcasts them.
Note that although $BroadcastBlockUpdate$ only manages $broadcastBlockUpdate$, it is general enough to be repurposed for any other channel that needs broadcasting.

To explain how $BroadcastBlockUpdate$ works we first need to introduce broadcast tasks.

\begin{align*}
    BroadcastTask &= (peers, data)
\end{align*}

A broadcast task represents some data that should be lifted to a list of peers.

$BroadcastBlockUpdate$ consists of two parts: $BlockUpdateSetup$ and $BlockUpdateExecute$

\begin{align*}
    BroadcastBlockUpdate &= BlockUpdateSetup \para BlockUpdateExecute
\end{align*}

The idea is to have $BlockUpdateSetup$ setup the broadcast tasks and have $BlockUpdateExecute$ execute them.

$BlockUpdateSetup$ inputs messages on $broadcastBlockUpdate$ and lifts a broadcast task with the known peers and the message data to $blockUpdateTasks$

\begin{align*}
    BlockUpdateSetup &=\ !broadcastBlockUpdate.(data).peers(localpeers).\\
    &\quad (peers\lift{}{localpeers}\ |\ blockUpdateTasks\lift{}{(localpeers,data)})
\end{align*}

$BlockUpdateExecute$ inputs tasks on $blockUpdateTasks$. When a task is received, it lifts a message on the $blockUpdate$ channel of the head of the task peers with the task message data. If the tail of the task peers is non-empty, it also lifts the remainder of the task to $blockUpdateTasks$, that is a broadcast task with the tail of the task peers and the message data.

\begin{align*}
    BlockUpdateExecute &=\ !blockUpdateTasks(task).(blockUpdateOf(head(peersOf(task)))\\
    &\quad \lift{}{dataOf(task)}\ |\ [tail(peersOf(task))=emptyList,\nil,\\
    &\quad blockUpdateTasks\lift{}{(tail(peersOf(task)),dataOf(task))})
\end{align*}


%consensus for the blockchain
%secure transactions
%encrytped authentication
%digital signatures
%public key encryption
%distributed ledgers based on consensus
%smart contracts
%fault tolerant transactions processing
%zero-knowledge proofs

\section{Security keywords}

\subsection{Distributed fault tolerance}
Since the blockchain protocol is a distributed database, then the protocol must also contain fault tolerance regarding problems like, a node in the network dies or is attacked, a miner encounters an error during a mining process.
%lidt længere og mere hvorfor det her er skide vigtigt

\subsection{Immutability}
The blockchain is all about having a database that has a unmodified history, which it also uses to verify new blocks, therefor the blockchain or rather the blocks in the chain must be designed to be immutable. The blocks must never be modified after creation, which is very relevant for use cases of currency. Changing blocks would mean that users could change already made transactions and thereby use already used money which would cause serious issues.
%skal have en bedre start, og 1 eller 2 argumenter ekstra

\subsection{Consensus}
Consensus is the process by which the protocol, or rather the nodes in the network agrees if the new block is valid or not, and is allowed to be added to the chain. The popular choices of consensus protocols are proof-of-stake and proof-of-work. The proof-of-work protocol is proved to be valid, but the proof-of-stake is still under evaluation, and therefore relevant to analyze. New protocols for achieving consensus are quickly appearing, and can therefore be relevant to consider to analyze and verify them.\cite{web:consensus}
%lidt rodet men kan skal bare lige rydes op

\subsection{Authenticity}
Authenticity is very important for databases, since user rights are often used for different access rights. Zero knowledge proof is a way to verify whether a user, or in blockchains case a node, is who it claims to be. The advantage with zero knowledge proof is that no information is shared. Zero knowledge proofs works as such:\\
A prover tries to prove his claim to a verifier. The verifier then has a challenge that the prover must be able to clear to convince the verifier that the provers claim is true. To clear the challenge the prover needs some information, which the verifier wants to confirm whether the prover has. The verifier does not actually want the information, that the prover claims to own since it is very important that the information is not leaked. If the prover clears the challenge, the verifier will ask the challenge once more and this will continue until the verifier is convinced that the provers claim is true.\cite{art:zero_knowledge_proofs}


%Nok bedre med en bedre forklaring

\subsection{Smart contracts}
A smart contrast is a way to both define contracts between parties and also automatically enforce the contracts obligations. Smart contracts are enforced by code and the entire blockchain is working to ensure the contract is enforced, so no single party can change the contract because of its distributed nature\cite{website:blockchaintechnologies}.

\subsection{Conclusion}
We will in look more upon the authentication of a system, because authentication is a major part of a blockchain protocol, which we will try to make a implementation of, and test whether it holds it authentication or not. For that we are going to need a small calculus for the implementation and a set of rules for the check of authentication.


\bibliography{../../Bibliography/Bib}
\end{document}\grid
\grid
