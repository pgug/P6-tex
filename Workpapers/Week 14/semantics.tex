\subsection{Structural Congruence}
The structural congruence of processes is denoted by the relation $\equiv$.

\begin{figure}[h]
    \begin{align*}
        P\para 0 \equiv\ &P \equiv 0\para P\\
        P\para Q &\equiv Q\para P\\
        P\para (Q\para R)&\equiv (P\para Q)\para R
    \end{align*}
    \caption{The rules of structural congruence for processes}
\end{figure}

%\begin{figure}[h]
%    \begin{center}
%        \begin{tabular}[c]{ll}
%            \runa{Equivalence} & \infrule{P\equiv P'\quad P'\ra Q'\quad Q'\equiv Q}{P\ra Q}
%        \end{tabular}
%    \end{center}
%    \caption{Structural Congruence for Equivalence}
%    \label{fig:equi}
%\end{figure}

\FloatBarrier

\subsection{Name Equivalence}
The equivalence of names is denoted by the relation $\equiv _N$

\begin{figure}[h]
	\begin{align}
	& \infrule{}{\quot{\drop{x}}\equiv _N x} \tag{Drop and Quote}
	\end{align}
	\caption{Name equivalence rule for drop and quote}
	\label{fig:dropquot}
\end{figure}

\noindent
The rule in \figref{fig:dropquot} states that if $x$ is dropped, and then quoted, then it should stay equivalent with $x$.

\begin{figure}[h]
	\begin{align}
	& \infrule{P\equiv Q}{\quot{P} \equiv _N \quot{Q}} \tag{Structural equivalence}
	\end{align}
	\caption{Name equivalence rule for structural congruence}
	\label{fig:strucequiv}
\end{figure}

\noindent
The rule in \figref{fig:strucequiv} states, that if $P$ and $Q$ are structural congruent with each other, then the quoted $P$ and $Q$ are name equivalent with each other.

\FloatBarrier

\subsection{Reduction Rules}

\begin{figure}[h]
	\begin{align}
	& \infrule{x_0 \equiv_N x_1}{\lift{x_0}{M}\para\inp{x_1}{y}P\ra P\{M/y\}} \tag{Comm}
	\end{align}
	\caption{Communication rule for when a term is lifted instead of a process.}
	\label{fig:com}
\end{figure}

%\begin{figure}[h]
%    \begin{center}
%        \begin{tabular}[c]{ll}
%            \runa{Comm} & \infrule{x_0 \equiv_N x_1}{\lift{x_0}{M}\para\inp{x_1}{y}P\ra P\{M/y\}}
%        \end{tabular}
%    \end{center}
%    \caption{Communication rule for when a \textit{Term} is lifted instead of a \textit{Process}.}
%    \label{fig:com}
%\end{figure}
%\noindent

\begin{figure}[!h]
	\begin{align}
	& \infrule{P\ra P'}{P\para Q\ra P'\para Q} \tag{Parallel}
	\end{align}
	\caption{Reduction rule for parallel processes}
	\label{fig:para}
\end{figure}

\begin{figure}[h]
	\begin{align}
		& \infrule{P\equiv P'\quad P'\ra Q'\quad Q'\equiv Q}{P\ra Q} \tag{Equivalence}
	\end{align}
	\caption{Reduction rule for structural congruence}
	\label{fig:equi}
\end{figure}

%\begin{figure}[h]
%    \begin{center}
%        \begin{tabular}[c]{ll}
%            \runa{Drop and Quote} & \infrule{}{\quot{\drop{x}}\equiv _N x}
%        \end{tabular}
%    \end{center}
%    \caption{Reduction rule for Drop and Quote}
%    \label{fig:dropquot}
%\end{figure}
%\noindent

%\begin{figure}[h]
%    \begin{center}
%        \begin{tabular}[c]{ll}
%            \runa{Structural equivalence} & \infrule{P\equiv Q}{\quot{P} \equiv _N \quot{Q}}
%        \end{tabular}
%    \end{center}
%    \caption{Reduction rule for Structural equivalence}
%    \label{fig:strucequiv}
%\end{figure}
%\noindent

%\begin{figure}[!h]
%    \begin{center}
%        \begin{tabular}[c]{ll}
%            \runa{Parallel} & \infrule{P\ra P'}{P\para Q\ra P'\para Q}
%        \end{tabular}
%    \end{center}
%    \caption{Reduction rule for Parallel}
%    \label{fig:para}
%\end{figure}
%\noindent

\FloatBarrier
