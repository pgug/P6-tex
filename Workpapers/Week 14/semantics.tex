\subsection{Structural Congruence}
\begin{figure}[h]
    \begin{align*}
        P\para 0 \equiv &P \equiv 0\para P\\
        P\para Q &\equiv Q\para P\\
        P\para (Q\para R)&\equiv (P\para Q)\para R
    \end{align*}
    \caption{The structural congruence for parallel process.}
\end{figure}

%\begin{figure}[h]
%    \begin{center}
%        \begin{tabular}[c]{ll}
%            \runa{Equivalence} & \infrule{P\equiv P'\quad P'\ra Q'\quad Q'\equiv Q}{P\ra Q}
%        \end{tabular}
%    \end{center}
%    \caption{Structural Congruence for Equivalence}
%    \label{fig:equi}
%\end{figure}

\begin{figure}[h]
    \begin{align}
        & \infrule{P\equiv P'\quad P'\ra Q'\quad Q'\equiv Q}{P\ra Q} \tag{Equivalence}
    \end{align}
    \caption{Structural Congruence for Equivalence}
    \label{fig:equi}
\end{figure}
\noindent
The structural equivalence for $P$ and $Q$, as seen in \figref{fig:equi}
\FloatBarrier

\subsection{Reduction Rules}

%\begin{figure}[h]
%    \begin{center}
%        \begin{tabular}[c]{ll}
%            \runa{Comm} & \infrule{x_0 \equiv_N x_1}{\lift{x_0}{M}\para\inp{x_1}{y}P\ra P\{M/y\}}
%        \end{tabular}
%    \end{center}
%    \caption{Communication rule for when a \textit{Term} is lifted instead of a \textit{Process}.}
%    \label{fig:com}
%\end{figure}
%\noindent

\begin{figure}[h]
    \begin{align}
        & \infrule{x_0 \equiv_N x_1}{\lift{x_0}{M}\para\inp{x_1}{y}P\ra P\{M/y\}} \tag{Comm}
    \end{align}
    \caption{Communication rule for when a \textit{Term} is lifted instead of a \textit{Process}.}
    \label{fig:com}
\end{figure}
\noindent
The communication rule for input and output is seen in \figref{fig:com}.

%\begin{figure}[h]
%    \begin{center}
%        \begin{tabular}[c]{ll}
%            \runa{Drop and Quote} & \infrule{}{\quot{\drop{x}}\equiv _N x}
%        \end{tabular}
%    \end{center}
%    \caption{Reduction rule for Drop and Quote}
%    \label{fig:dropquot}
%\end{figure}
%\noindent

\begin{figure}[h]
    \begin{align}
        & \infrule{}{\quot{\drop{x}}\equiv _N x} \tag{Drop and Quote}
    \end{align}
    \caption{Reduction rule for Drop and Quote}
    \label{fig:dropquot}
\end{figure}
\noindent
The drop and quote rules are applied to $x$ in \figref{fig:dropquot}. Where if $x$ if first dropped, and then quoted then it should be name equivalent with $x$.

%\begin{figure}[h]
%    \begin{center}
%        \begin{tabular}[c]{ll}
%            \runa{Structural equivalence} & \infrule{P\equiv Q}{\quot{P} \equiv _N \quot{Q}}
%        \end{tabular}
%    \end{center}
%    \caption{Reduction rule for Structural equivalence}
%    \label{fig:strucequiv}
%\end{figure}
%\noindent

\begin{figure}[h]
    \begin{align}
        & \infrule{P\equiv Q}{\quot{P} \equiv _N \quot{Q}} \tag{Structural equivalence}
    \end{align}
    \caption{Reduction rule for Structural equivalence}
    \label{fig:strucequiv}
\end{figure}
\noindent
The structural equivalence of $P$ and $Q$, where if $P$ and $Q$ are congruent with each other, the quoted version of $P$ and $Q$ are name equivalent with each other, see \figref{fig:strucequiv}.

%\begin{figure}[!h]
%    \begin{center}
%        \begin{tabular}[c]{ll}
%            \runa{Parallel} & \infrule{P\ra P'}{P\para Q\ra P'\para Q}
%        \end{tabular}
%    \end{center}
%    \caption{Reduction rule for Parallel}
%    \label{fig:para}
%\end{figure}
%\noindent

\begin{figure}[!h]
    \begin{align}
        & \infrule{P\ra P'}{P\para Q\ra P'\para Q} \tag{Parallel}
    \end{align}
    \caption{Reduction rule for Parallel}
    \label{fig:para}
\end{figure}
\noindent
The reduction rule for $P$ parallel with $Q$ as seen in \figref{fig:para}.
\FloatBarrier
