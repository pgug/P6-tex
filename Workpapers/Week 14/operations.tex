\subsubsection{Encryption and Decryption}
The operation of encryption and decryption is each others inverse.
\begin{figure}[h]
    \begin{align*}
        &dec(enc(M_1, M_2),M_2) = M_1 \tag{Decrypt}
    \end{align*}
    \caption{Operations for encrypting and decrypting a message $M_1\ \mathrm{with\ key}\ M_2$.}
\end{figure}
\FloatBarrier

\subsubsection{Compare operation}
The comparison operations runs the process \textit{P} if the statement holds true.
\begin{figure}[h]
    \begin{align}
        &[M_1 = M_2]P \ra P \tag{Equal}\\
        &[M_1 \neq M_2]P \ra P \tag{NotEqual}\\
        &[M_1 > M_2]P \ra P \tag{Greater}\\
        &[M_1 \geq M_2]P \ra P \tag{GreaterEqual}\\
        &[M_1 < M_2]P \ra P \tag{Lesser}\\
        &[M_1 \leq M_2]P \ra P \tag{LesserEqual}
    \end{align}
    \caption{Listed are the evaluation operations. All evaluation operations evaluate $M_1\ \mathrm{against}\ M_2$ on some process \textit{P}, resulting in \textit{P} if the evaluation holds true.}
\end{figure}

\FloatBarrier

\subsubsection{List operations}

The list operations allows the abstraction of lists. They are described as follows:

\begin{description}
	\item[head(list)] The first element of the list
	\item[tail(list)] The list without its first element
	\item[append(x, list)] The list with element x appended to its beginning.
\end{description}

These operations are defined by the rules of \figref{listoprules}.

\begin{figure}[h]
	\begin{align*}
		&head(append(M_1, M_2)) = M_1 \tag{Head} \\
		&tail(append(M_1, M_2)) = M_2 \tag{Tail}
	\end{align*}
	\caption{The rules of the list operations}
	\label{listoprules}
\end{figure}

\FloatBarrier


