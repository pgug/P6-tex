\subsection{Reduction Example}
The replication can run \textit{P} parallel with \textit{!P} infinity many times.
%\begin{figure}[h]
%    \begin{center}
%        \begin{tabular}[c]{cll}
%            & !P & \runa{Initial} \\

%            $\equiv$ & \lift{x}{\inp{x}{y}(\lift{x}{\drop{y}}\para\drop{y})\para P}\para\inp{x}{y}(\lift{x}{\drop{y}}\para \drop{y}) & \runa{Substitution} \\

%            $\ra$ & \lift{x}{\drop{\quot{\inp{x}{y}(\lift{x}{\drop{y}}\para\drop{y})\para P}}}\para\drop{\quot{\inp{x}{y}(\lift{x}{\drop{y}}\para\drop{y})\para P}} & \runa{Communication} \\

%            $\equiv$ & \lift{x}{\inp{x}{y}(\lift{x}{\drop{y}}\para\drop{y})\para P}\para\inp{x}{y}(\lift{x}{\drop{y}}\para \drop{y})\para P & \runa{DropQuote} \\

%            $\equiv$ & !P\para P & \runa{Substitution}
%        \end{tabular}
%    \end{center}
%    \caption{Reduction example of replication}
%    \label{fig:reductionexample}
%\end{figure}


\begin{figure}[h]
    \begin{align}
        &!P \tag{Initial} \\
        &\equiv \lift{x}{\inp{x}{y}(\lift{x}{\drop{y}}\para\drop{y})\para P}\para\inp{x}{y}(\lift{x}{\drop{y}}\para \drop{y}) \tag{Substitution} \\
        &\ra \lift{x}{\drop{\quot{\inp{x}{y}(\lift{x}{\drop{y}}\para\drop{y})\para P}}}\para\drop{\quot{\inp{x}{y}(\lift{x}{\drop{y}}\para\drop{y})\para P}} \tag{Comm}\\
        &\equiv \lift{x}{\inp{x}{y}(\lift{x}{\drop{y}}\para\drop{y})\para P}\para\inp{x}{y}(\lift{x}{\drop{y}}\para \drop{y})\para P \tag{DropQuote} \\
        &\equiv !P\para P \tag{Substitution}
    \end{align}
    \caption{Reduction example of replication}
    \label{fig:reductionexample}
\end{figure}


\noindent
In \figref{fig:reductionexample} we have by reduction proved that $!P$ can run $P$ in parallel infinity many times, by applying the rule again. The Replication is an coding developed by  L. G. Meredith and M. Radestock in \citep{Meredith2005}
