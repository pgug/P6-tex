%Doc settings
\documentclass[12pt,a4paper]{article}
\usepackage[utf8]{inputenc}
\usepackage[english]{babel}
\usepackage{lmodern} %vector font Latin Modern
\usepackage[T1]{fontenc} %font encoding
\usepackage{amsmath}
\usepackage{amssymb}
\usepackage{stmaryrd}

% Bibliography
\usepackage{natbib}
\setcitestyle{numbers,square}
\bibliographystyle{unsrtnat}
% Proceskonstruktioner

\newcommand{\quot}[1]{\ensuremath{\ulcorner  #1  \urcorner}}
\newcommand{\drop}[1]{\ensuremath{\urcorner  #1  \ulcorner}}
\newcommand{\lift}[2]{\ensuremath{#1 \langle \! |  #2  | \! \rangle}}
\newcommand{\nil}{\ensuremath{\mathbf{0}}}

\newcommand{\inp}[2]{\ensuremath{#1(#2).}}
\newcommand{\para}{\ensuremath{\mid}}

% Inferensregler og regelnavne

\newcommand{\infrule}[2]  {\parbox{4.5cm}{$$ \frac{#1}{#2}\hspace{.5cm}$$}}
\newcommand{\runa}[1]{\textsc{(#1})}

% Pile

\newcommand{\ra}{\rightarrow}

% Henvisninger

\newcommand{\tabref}[1]{Table \ref{#1}}


\newcommand{\redchain}[3]{
\begin{table}[!h]
\begin{center}
\begin{tabular}{cll}
$$#1$$
\end{tabular}
\end{center}
\caption{#2}
\label{#3}
\end{table}
}

\author{Group D608F16}
\title{Rho-calculus Constructions}


\begin{document}
\maketitle

\section{Workpaper week 10}
The notation of the $\rho$-Calculus.\citep{Meredith2005}\\
P,Q ::= \nil \ Nil\\
|\inp{x}{y}P Input\\
|\lift{x}{P} Lift\\
|\drop{x} Drop\\
|P\para Q Parallel\\
x,y::= \quot{P} Quot\\


\subsection{Numbers}
This example could be a way of expressing numbers in $\rho$-Calculus. Where the number takes place as $x$ or $y$ in the notation. By quoting the last number you get a higher number, and by dropping a number you get a lower number, and at 0 you get the process P.\\
$0:$ \quot{\nil}\\
$1:$ \quot{\lift{\quot{\nil}}{\nil}}\\
$2:$ \quot{\lift{\quot{\nil}}{\lift{\quot{\nil}}{\nil}}}\\
$3:$ \quot{\lift{\quot{\nil}}{\lift{\quot{\nil}}{\lift{\quot{\nil}}{\nil}}}}\\
$4:$ \quot{\lift{\quot{\nil}}{\lift{\quot{\nil}}{\lift{\quot{\nil}}{\lift{\quot{\nil}}{\nil}}}}}\\

\subsection{Structure}
We can express some of the programming terms in $\rho$-Calculus, but more has to be discovered.\\
$if\ a == b\ then\ P$ Can be expressed in $\rho$-Calculus as: \lift{a}{\nil} \para \inp{b}{x}P
\\ We send \nil \ on channel $a$ and if channel $a$ and channel $b$ is equal then channel $b$ will get a \nil \ and execute $P$.
%$[P,Q]:$ \lift{x}{\drop{P \para Q}} \para \inp{x}{y}\quot{y}
\\\\
$P+Q$ Can be express in $\rho$-Calculus as: \lift{a}{0} \para \inp{a}{x}P \para \inp{a}{x}Q
\\We can make a deterministic choice between $P$ and $Q$ by sending \nil \ on channel $a$ and in parallel two channel $a$ tries both to receive, but only one would succeed and would run its process $P$ or process $Q$.

\subsection{Yet to be solved}
We need to express more programming structures, before we can use the $\rho$-Calculus to model a blockchain structure or even r-chain. These are a few examples of what we need to express, and more should follow.
\begin{itemize}
\item if a$\neq$ b then P
\item if a<b then P
\item if a>b then P
\item if a==b then P else Q
\end{itemize}


\bibliography{../../Bibliography/Bib}
\end{document}
