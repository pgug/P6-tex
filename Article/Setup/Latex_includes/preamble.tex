%Doc settings
\documentclass[12pt,twoside,a4paper%,openright
]{report}
\usepackage[utf8]{inputenc}
\usepackage[english]{babel}
\usepackage{lmodern} %vector font Latin Modern
\usepackage[T1]{fontenc} %font encoding

%Tabels, color, figures and alike
\usepackage{colortbl}
\usepackage[dvipsnames]{xcolor}
\usepackage{float}
\usepackage{graphicx}
\usepackage{tabularx}
\usepackage{rotating}
\usepackage{caption}
\usepackage[nottoc]{tocbibind}
\captionsetup{%
  font=footnotesize,% set font size to footnotesize
  labelfont=bf % bold label (e.g., Figure 3.2) font
}
\usepackage{pdfpages}
%\usepackage{wrapfig} %Wrapping graphics

%fixme
%\usepackage[final,inline,nomargin,index]{fixme} %draft

%\fxsetup{theme=color} %+mode=multiuser
%\FXRegisterAuthor{sv}{asv}{Me}
%\FXRegisterAuthor{aa}{aaa}{Other}
%See http://tex.stackexchange.com/questions/99904/is-there-a-way-to-assign-different-colors-to-different-authors-in-fixme

%comments
\newcommand\todo[1]{\textcolor{red}{?? #1}}
%remove comments
%\renewcommand\todo[1]{}

%Math
\usepackage{amsmath}
\usepackage{amssymb}

%Layout
\usepackage[
  inner=28mm,% left margin on an odd page
  outer=28mm,% right margin on an odd page
  ]{geometry}

\usepackage{verbatim}


% Clear empty pages between chapters
\let\origdoublepage\cleardoublepage
\newcommand{\clearemptydoublepage}{%
  \clearpage
  {\pagestyle{empty}\origdoublepage}%
}
\let\cleardoublepage\clearemptydoublepage

% Change the headers and footers
\usepackage{fancyhdr}
\pagestyle{fancy}
\fancypagestyle{plain}{ %
  \fancyhf{} % remove everything
  \renewcommand{\headrulewidth}{0pt} % remove lines as well
  \renewcommand{\footrulewidth}{0pt}
  \rfoot{\fancyplain{}{\thepage}}
  \lfoot{\fancyplain{}{D608F17}}
}
\fancyhf{} %delete everything
\renewcommand{\headrulewidth}{0pt} %remove the horizontal line in the header
%\fancyfoot[C]{Page \thepage \ of \pageref{LastPage}} %page number on all pages
\setlength{\headheight}{15pt} 
\renewcommand{\chaptermark}[1]{ \markboth{#1}{} }
\renewcommand{\sectionmark}[1]{ \markright{#1}{} }
\rfoot{\thepage}
\lfoot{D504E16}
\lhead{\thechapter \enspace |\enspace \leftmark}

%\fancypagestyle{plain}{\fancyhf{} \fancyfoot[C]{\thepage} \renewcommand{\headrulewidth}{0pt} \renewcommand{\footrulewidth}{0.0pt}}


% Do not stretch the content of a page. Instead,
% insert white space at the bottom of the page
\raggedbottom
% Enable arithmetics with length. Useful when
% typesetting the layout.
\usepackage{calc}


%Chapter layout
\newcommand{\hsp}{\hspace{20pt}}
\definecolor{gray75}{gray}{0.75}

\makeatletter
\renewcommand{\@makechapterhead}[1]{%
    \vspace*{50 pt}%
    {\setlength{\parindent}{0pt} \raggedright \normalfont
    \bfseries\Huge\thechapter\hsp{\color{gray75}|}\hsp #1
    \par\nobreak\vspace{40 pt}}
}
\makeatother


%\usepackage{titlesec}
%\definecolor{gray75}{gray}{0.75}
%\newcommand{\hsp}{\hspace{20pt}}
%\titleformat{\chapter}[hang]{\Huge\bfseries}{\thechapter\hsp\textcolor{gray75}{|}\hsp}{0pt}{\Huge\bfseries}



%¤¤ Afsnitsformatering ¤¤ %
%\setlength{\parindent}{0mm}           	
% Størrelse af indryk
%\setlength{\parskip}{3mm}          			
% Afstand mellem afsnit ved brug af double Enter
%\linespread{1,1}												
% Linie afstand



% Bibliography
\usepackage{natbib}
\setcitestyle{numbers,square}
\bibliographystyle{unsrtnat}


% Misc
%\usepackage[nottoc]{tocbibind}
\usepackage{hyperref}
\hypersetup{%
	plainpages=false,%
	bookmarksnumbered=true,%
	colorlinks,%
	citecolor=black,%
	filecolor=black,%
	linkcolor=black,%
	urlcolor=black,%
	pdfstartview=FitH%
}

%Code font - From http://tex.stackexchange.com/questions/124953/syntax-highlighting-in-listings-for-c-that-it-looks-like-in-visual-studio
\usepackage{listings}

\definecolor{grayC}{RGB}{250,250,250}
\definecolor{Dgray}{RGB}{200,200,200}
\definecolor{bluekeywords}{rgb}{0,0,1}
\definecolor{greencomments}{rgb}{0,0.5,0}
\definecolor{redstrings}{rgb}{0.64,0.08,0.08}
\definecolor{xmlcomments}{rgb}{0.5,0.5,0.5}
\definecolor{types}{rgb}{0.17,0.57,0.68}

\usepackage{listings}
\lstdefinestyle{C}{ %
  %backgroundcolor=\color{grayC},
  basicstyle=\ttfamily\small,        % the size of the fonts that are used for the code
  breakatwhitespace=true,         % sets if automatic breaks should only happen at whitespace
  breaklines=true,                 % sets automatic line breaking
  captionpos=b,                    % sets the caption-position to bottom
  commentstyle=\color{greencomments},    % comment style
  deletekeywords={...},            % if you want to delete keywords from the given language
  escapeinside={(*@}{@*)},          % if you want to add LaTeX within your code
  extendedchars=true,              % lets you use non-ASCII characters; for 8-bits encodings only, does not work with UTF-8
  frame=single,                    % adds a frame around the code
  keepspaces=true,                 %% keeps spaces in text, useful for keeping indentation of code (possibly needs columns=flexible)
  keywordstyle=\color{bluekeywords},       % keyword style
  language=[Sharp]C,              % the language of the code
  morekeywords={partial, var, value, get, set},            % if you want to add more keywords to the set
  numbers=left,                    % where to put the line-numbers; possible values are (none, left, right)
  numbersep=5pt,                   % how far the line-numbers are from the code
  numberstyle=\tiny\color{black},  % the style that is used for the line-numbers
  rulecolor=\color{Dgray},         % if not set, the frame-color may be changed on line-breaks within not-black text (e.g. comments (green here))
  showspaces=false,                % show spaces everywhere adding particular underscores; it overrides 'showstringspaces'
  showstringspaces=false,          % underline spaces within strings only
  showtabs=false,                  % show tabs within strings adding particular underscores
  stepnumber=1,                    % the step between two line-numbers. If it's 1, each line will be numbered
  stringstyle=\color{redstrings},     % string literal style
  tabsize=2,                       % sets default tabsize to 2 spaces
  title=\lstname                   % show the filename of files included with \lstinputlisting; also try caption instead of title
}

%colors for C#:
\definecolor{MylightGray}{RGB}{210,210,210}
\definecolor{bluekeywords}{rgb}{0.13,0.13,1}
\definecolor{greencomments}{rgb}{0,0.5,0}
\definecolor{redstrings}{rgb}{0.9,0,0}
\definecolor{Cyan}{rgb}{0.0,0.6,0.6}
\definecolor{LimeGreen}{RGB}{205,133,0}
\definecolor{LineNumbers}{RGB}{0,0,0}

%colors for XML:
\definecolor{forestgreen}{RGB}{34,139,34}
\definecolor{orangered}{RGB}{239,134,64}
\definecolor{darkblue}{rgb}{0.0,0.0,0.6}
\definecolor{gray}{RGB}{210,210,210}
\definecolor{purple}{RGB}{148,0,211}

%colors for language definition
\definecolor{editorLightGray}{cmyk}{0.05, 0.05, 0.05, 0.1}
\definecolor{editorGray}{RGB}{210,210,210}
\definecolor{editorPurple}{cmyk}{0.5, 1, 0, 0}
\definecolor{editorWhite}{cmyk}{0, 0, 0, 0}
\definecolor{editorBlack}{cmyk}{1, 1, 1, 1}
\definecolor{editorOrange}{cmyk}{0, 0.8, 1, 0}
\definecolor{editorBlue}{cmyk}{1, 0.6, 0, 0}
\definecolor{editorPink}{cmyk}{0, 1, 0, 0}

%colors JS colors
\definecolor{lightgray}{RGB}{210,210,210}
\definecolor{darkgray}{rgb}{.4,.4,.4}
\definecolor{purple}{rgb}{0.65, 0.12, 0.82}

%colors Table colors
\definecolor{TblYellow}{RGB}{252,232,178}
\definecolor{TblBlue}{RGB}{201,218,248}
\definecolor{TblGreen}{RGB}{183,225,205}

%%%%%%%%%%%%%%%%%%%%%%%%%%%%%%%%%%%%%%%%
%defining new languages
%%%%%%%%%%%%%%%%%%%%%%%%%%%%%%%%%%%%%%%%
% CSS
\lstdefinelanguage{CSS}{
    keywords={color:,background-image:,margin:,margin-top:,margin-bottom:,margin-right:,margin-left:,padding,font,weight,display,position:,top:,left:,right:,bottom:,list:,style:,border:,border-radius:,size:,white:,space:,min:,width:, transition:, transform:, transition-property:, transition-duration:, transition-timing-function:}, 
    sensitive=true,
    morecomment=[l]{//},
    morecomment=[s]{/*}{*/},
    morestring=[b]',
    morestring=[b]",
    alsoletter={:,.},
    alsodigit={-}
}
% Javascript
\lstdefinelanguage{JavaScript}{
  keywords={typeof, new, true, false, catch, function, return, null, catch, switch, var, if, in, while, do, else, case, break},
  keywordstyle=\color{blue}\bfseries,
  ndkeywords={class, export, boolean, throw, implements, import, this},
  ndkeywordstyle=\color{blue}\bfseries,
  identifierstyle=\color{black},
  sensitive=false,
  comment=[l]{//},
  morecomment=[s]{/*}{*/},
  commentstyle=\color{greencomments}\ttfamily,
  stringstyle=\color{red}\ttfamily,
  morestring=[b]',
  morestring=[b]"
}

% eXtensible Markup Language
\lstdefinelanguage{XML} {
  morestring=[b]",
 morestring=[s][\color{black}]{>}{<},
  morecomment=[s]{<!--}{-->},
  stringstyle=\color{black},
  identifierstyle=\color{darkblue},
  keywordstyle=\color{cyan},
  morekeywords={xmlns,version,type}
}

%setup to C# in listings
\lstdefinestyle{C[Sharp]}{
    language=[Sharp]C,
    %backgroundcolor=\color{MylightGray},
    frame=single,
    showspaces=false,
    showtabs=false,
    stepnumber = 1,
    tabsize = 2,
    breaklines=true,
    postbreak=\raisebox{0ex}[0ex][0ex]{\ensuremath{\color{red}\hookrightarrow\space}},
    showstringspaces=false,
    breakatwhitespace=false,
    escapeinside={(*@}{@*)},
    commentstyle=\color{greencomments},
    morecomment=[l]{//}, %use comment-line-style!
    morecomment=[s]{/*}{*/}, %for multi-line comments
    keywordstyle=\color{bluekeywords}\bfseries,
    stringstyle=\color{redstrings},
    basicstyle=\ttfamily\small,
    numbers=left,                    % where to put the line-numbers; possible values are (none, left, right)
    numbersep=5pt,                   % how far the line-numbers are from the code
    numberstyle=\tiny\color{LineNumbers},
    literate=%
    {æ}{{\ae}}1
    {å}{{\aa}}1
    {ø}{{\o}}1
    {Æ}{{\AE}}1
    {Å}{{\AA}}1
    {Ø}{{\O}}1,
    extendedchars = \true,
    inputencoding=ansinew,
    morekeywords={
    abstract, event, new, struct,as, explicit, null, switch,base, extern, object, this,bool, false, operator, throw,break, finally, out, true,byte, fixed, override, try,case, float, params, typeof,catch, for, private, uint,char, foreach, protected, ulong,checked, goto, public, unchecked,class, if, readonly, unsafe,const, implicit, ref, ushort,continue, in, return, using,decimal, int, sbyte, virtual,default, interface, sealed, volatile,delegate, internal, short, void,do, is, sizeof, while,double, lock, stackalloc,else, long, static,enum, namespace, string, Convert, XmlAttribute, String,Program,
    File,Regex,Exception,BackgroundWorker,Dictionary,MainWindow, FileRead,List,var,},
    keywords=[2]{Label,Price, Location}, keywordstyle=[2]\color{blue},%tilføj ting der skal highlightes samme farve som klasser!
    keywords=[3]{IinterfaceHere }, keywordstyle=[3]\color{LimeGreen},%tilføj interfaces
}
%defining style for HTML
\lstdefinestyle{HTML}{%
    language=html,
    %backgroundcolor=\color{MylightGray},
    frame=single,
    showspaces=false,
    showtabs=false,
    stepnumber = 1,
    tabsize = 2,
    breaklines=true,
    postbreak=\raisebox{0ex}[0ex][0ex]{\ensuremath{\color{red}\hookrightarrow\space}},
    showstringspaces=false,
    breakatwhitespace=false,
    tagstyle=\color{black},
    basicstyle=\small\ttfamily,
    keywordstyle=\color{blue},
    commentstyle=\color{greencomments},
    stringstyle=\color{red},
    numbers=left,                    % where to put the line-numbers; possible values are (none, left, right)
    numbersep=5pt,                   % how far the line-numbers are from the code
    numberstyle=\tiny\color{LineNumbers},
}

% XML
\lstdefinestyle{XML}{
    language=XML,
    %backgroundcolor = \color{MylightGray},
    frame = single,
    showspaces = false,
    showtabs = false,
    stepnumber = 1,
    tabsize = 2,
    breaklines = true,
    postbreak=\raisebox{0ex}[0ex][0ex]{\ensuremath{\color{red}\hookrightarrow\space}},
    showstringspaces = false,
    breakatwhitespace = false,
    tagstyle = \color{black},
    basicstyle = {\small\ttfamily},
    keywordstyle = \color{blue},
    commentstyle = \color{greencomments},
    stringstyle = \color{red},
    numbers=left,                    % where to put the line-numbers; possible values are (none, left, right)
    numbersep=5pt,                   % how far the line-numbers are from the code
    numberstyle=\tiny\color{LineNumbers},
}

%defining style for CSS
\lstdefinestyle{Css}{%
    language = CSS,
    %backgroundcolor=\color{MylightGray},
    frame=single,
    showspaces=false,
    showtabs=false,
    stepnumber = 1,
    tabsize = 2,
    breaklines=true,
    postbreak=\raisebox{0ex}[0ex][0ex]{\ensuremath{\color{red}\hookrightarrow\space}},
    showstringspaces=false,
    breakatwhitespace=false,
    escapeinside={(*@}{@*)},
    basicstyle=\ttfamily\small,
    commentstyle=\color{greencomments},
    keywordstyle=\color{blue},
    numbers=left, 
    numbersep=5pt,                   % how far the line-numbers are from the code
    numberstyle=\tiny\color{LineNumbers},
    literate=%
    {æ}{{\ae}}1
    {å}{{\aa}}1
    {ø}{{\o}}1
    {Æ}{{\AE}}1
    {Å}{{\AA}}1
    {Ø}{{\O}}1,
    extendedchars = \true
}
%%%%%%%%%%%%%%%%%%%%%%%%%%%%%%
%JavaScript styling
%%%%%%%%%%%%%%%%%%%%%%%%%%%%%%
\lstdefinestyle{JS}{
   language=JavaScript,
   %backgroundcolor=\color{lightgray},
   frame=single,
   extendedchars=true,
   basicstyle=\small\ttfamily,
   showstringspaces=false,
   showspaces=false,
   numbers=left,
   numberstyle=\footnotesize,
   numbersep=9pt,
   tabsize=2,
   breaklines=true,
   showtabs=false,
   captionpos=b
}
\lstset{style = C[Sharp]}

%Defines and custom settingsg
\definecolor{aaublue}{RGB}{33,26,82}% dark blue
\definecolor{LightBlue}{RGB}{0,235,235}% Light blue mucho

%\setcounter{secnumdepth}{4}% Show down to subsubsection
	
%\setcounter{tocdepth}{2}
%\usepackage{here}					% Gør det muligt at placere figurer hvor du vil. \begin{figure}[!h] % Will not be floating.
%\usepackage{floatflt}				% Indsættelse af tabeller omsvøbt af tekst.
\usepackage{epstopdf}

%hypernation
\hyphenation{ex-am-ple hy-phen-a-tion short}
\hyphenation{long la-tex}


\usepackage{pdfpages}


%\usepackage{algpseudocode}


\newcommand{\aautitlepage}[3]{%
  {
    %set up various length
    \ifx\titlepageleftcolumnwidth\undefined
      \newlength{\titlepageleftcolumnwidth}
      \newlength{\titlepagerightcolumnwidth}
    \fi
    \setlength{\titlepageleftcolumnwidth}{0.5\textwidth-\tabcolsep}
    \setlength{\titlepagerightcolumnwidth}{\textwidth-2\tabcolsep-\titlepageleftcolumnwidth}
    %create title page
    \thispagestyle{empty}
    \noindent%
    \begin{tabular}{@{}ll@{}}
      \parbox{\titlepageleftcolumnwidth}{
        {%
          \includegraphics[width=\titlepageleftcolumnwidth]{Figures/aau_logo_en}
        }
      } &
      \parbox{\titlepagerightcolumnwidth}{\raggedleft\sf\small
        #2
      }\bigskip\\
       #1 &
      \parbox[t]{\titlepagerightcolumnwidth}{%
      \textbf{Abstract:}\bigskip\par
        \fbox{\parbox{\titlepagerightcolumnwidth-2\fboxsep-2\fboxrule}{%
          #3
        }}
      }\\
    \end{tabular}
    \vfill
    {%
      \noindent{\footnotesize\emph{The content of this report is freely available, but publication (with reference) may only be pursued due to agreement with the author.}}
    }
    \clearpage
  }
}

%Create english project info
\newcommand{\englishprojectinfo}[8]{%
  \parbox[t]{\titlepageleftcolumnwidth}{
    \textbf{Title:}\\ #1\bigskip\par
    \textbf{Theme:}\\ #2\bigskip\par
    \textbf{Project Period:}\\\small #3\bigskip\par
    \textbf{Project Group:}\\ \small #4\bigskip\par
    \textbf{Participants:}\\ #5\bigskip\par
    \textbf{Supervisor:}\\ #6\bigskip\par
    \textbf{Copies:} #7\bigskip\par
    \textbf{Number of Pages:} \pageref{LastPage}\bigskip\par
    \textbf{Date of Completion:}\\ #8
  }
}

%Create danish project info
\newcommand{\danishprojectinfo}[8]{%
  \parbox[t]{\titlepageleftcolumnwidth}{
    \textbf{Titel:}\\ #1\bigskip\par
    \textbf{Tema:}\\ #2\bigskip\par
    \textbf{Projektperiode:}\\ #3\bigskip\par
    \textbf{Projektgruppe:}\\ #4\bigskip\par
    \textbf{Deltager(e):}\\ #5\bigskip\par
    \textbf{Vejleder(e):}\\ #6\bigskip\par
    \textbf{Oplagstal:} #7\bigskip\par
    \textbf{Sidetal:} \pageref{LastPage}\bigskip\par
    \textbf{Afleveringsdato:}\\ #8
  }
}


%%%%%%%%%%%%%%%%%%%%%%%%%%%%%%%%%%%%%%%%%%%%%%%%
%Mængdebyggernotation
%%%%%%%%%%%%%%%%%%%%%%%%%%%%%%%%%%%%%%%%%%%%%%%%
\newcommand{\suchthat}{\ensuremath{\,\middle|\,}}
\newcommand{\set}[1]{\ensuremath{\left\{#1\right\}}}
\newcommand{\buildset}[2]{\ensuremath{\set{#1\suchthat#2}}}

\newcommand{\R}{{\mathbb{R}}}
\newcommand{\Z}{{\mathbb{Z}}}
\DeclareMathOperator*{\argmin}{\arg\!\min}
\delimitershortfall-1sp
\newcommand\abs[1]{\left\mid#1\right\mid}

%%%%%%%%%%%%%%%%%%%%%%%%%%%%%%%%%%%%%%%%%%%%%%%%
% simpler backslash
%%%%%%%%%%%%%%%%%%%%%%%%%%%%%%%%%%%%%%%%%%%%%%%%
\newcommand{\bs}{\textbackslash}

%%%%%%%%%%%%%%%%%%%%%%%%%%%%%%%%%%%%%%%%%%%%%%%%
% referencing commands
%%%%%%%%%%%%%%%%%%%%%%%%%%%%%%%%%%%%%%%%%%%%%%%%
\newcommand{\figref}[1]{[Figure \ref{#1}]}
