\subsection{Replication}
We also make use of replication, denoted by $!P$, to express inexhaustible processes.
The replication $!P$ proceeds as an arbitrary number of copies of $P$ in parallel. This is shown by the reduction example below. The replication was developed by L. G. Meredith and M. Radestock\citep{Meredith2005}.

\begin{align}
        &!P \tag{Initial} \\
        &\equiv \lift{x}{\inp{x}{y}(\lift{x}{\drop{y}}\para\drop{y})\para P}\para\inp{x}{y}(\lift{x}{\drop{y}}\para \drop{y}) \tag{Substitution} \\
        &\ra \lift{x}{\drop{\quot{\inp{x}{y}(\lift{x}{\drop{y}}\para\drop{y})\para P}}}\para\drop{\quot{\inp{x}{y}(\lift{x}{\drop{y}}\para\drop{y})\para P}} \tag{Comm}\\
        &\equiv \lift{x}{\inp{x}{y}(\lift{x}{\drop{y}}\para\drop{y})\para P}\para\inp{x}{y}(\lift{x}{\drop{y}}\para \drop{y})\para P \tag{DropQuote} \\
        &\equiv\ 
        !P\para P \tag{Substitution}
\end{align}

\FloatBarrier