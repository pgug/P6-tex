The syntax of the rho-calculus is described as follows.

\subsubsection{Processes}
\begin{description}
\item[Nil] Does nothing.
\item[Input] Receives a term lifted on channel \textit{x} and proceeds as process \textit{P} with the term replacing \textit{y}.
\item[Lift] Lifts a term \textit{M} on channel \textit{x}.
\item[Drop] Drops a name \textit{x}, so it becomes a process, and proceeds as the resulting process.
\item[Parallel] Proceeds as both process \textit{P} and process \textit{Q} in parallel.
\end{description}


\subsubsection{Names}
\begin{description}
\item[Quote] Quotes a process \textit{P}, so it becomes a name.
\end{description}

%\subsubsection{Boolean Arithmetic}

