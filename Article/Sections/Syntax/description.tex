The syntax of the rho-calculus is described as follows.\\

%\subsubsection{Processes}
%\begin{description}
%\item[Nil] Does nothing.
%\item[Input] Receives a term lifted on channel \textit{x} and proceeds as process \textit{P} with the term replacing \textit{y}.
%\item[Lift] Lifts a term \textit{M} on channel \textit{x}.
%\item[Drop] Drops a name \textit{x}, so it becomes a process, and proceeds as the resulting process.
%\item[Parallel] Proceeds as both process \textit{P} and process \textit{Q} in parallel.
%\end{description}

%\subsubsection{Names}
%\begin{description}
%\item[Quote] Quotes a process \textit{P}, so it becomes a name.
%\end{description}

%\subsubsection{Boolean Arithmetic}

%%%%%%%%%%%%%%%%%%%%%%%%%%%%%%%%%%%%%%%%%%%%%%%%%%%%%%%%%%%%%%%%%%%%%%%%%%%%%%%%%%%%%%%%%%%%%%%%%%%%%%%%%%%%%%%
%%%%%%%%%%%%%%%%%%%%%%%%%%%%%%%%%%%%% HERE IT IS WITH A NEW IMPROVED TASTE %%%%%%%%%%%%%%%%%%%%%%%%%%%%%%%%%%%%
%%%%%%%%%%%%%%%%%%%%%%%%%%%%%%%%%%%%%%%%%%%%%%%%%%%%%%%%%%%%%%%%%%%%%%%%%%%%%%%%%%%%%%%%%%%%%%%%%%%%%%%%%%%%%%%

\textbf{Processes}\\ % Yep I know it was a subsubsection, but LaTeX doesn't understand the hint that I want a linebreak.
The \textbf{Nil} process provides the solo atom every other process arise from. It relates to how 0 is the number from which all natural numbers are constructed from. The \textbf{Nil} process is a process that does nothing.\\
The \textbf{Input} operator is an operator that blocks the process until it receives an input.\\
\textbf{Lift} the lift operator allows a process to become a name. Since the language have a clear distension between names and processes therefor we need operations to convert between the two.\\
\textbf{Drop}, since you lift a process to make it a name, it is also necessarily to be able to convert the name to a process again. The \textbf{Drop} operator provides that functionality. \\
\textbf{Parallel} is a usual parallel composition, it executes two processes concurrently.\\

\textbf{Names}\\ % This shit is invisible until something is under it???
\textbf{Quote} Quotes a process \textit{P}, so it becomes a name. %% NOT READ AT ALL
\\\\
\textbf{Names and Process}\\
The relationship between names and process is a little tricky, because a name can be a dropped process and a process can be a quoted name. In \citep{Meredith2005} it is described how we build names out of the process $\nil$ by quoting it,$\quot{\nil}$ and to make more they send the quoted $\nil$ on its own channel and that can be dropped$\drop{\lift{\quot{\nil}}{\quot{\nil}}}$, because it is now a process. The new process can then be used as an channel and its message, and then again be dropped to make a new name. Even so they also describe in \citep{Meredith2005} how a $\nil$ parallel with $\nil$ can be dropped and that is becoming a name $\drop{\nil \para \nil}$