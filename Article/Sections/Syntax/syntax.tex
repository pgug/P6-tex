\section{Rho-Calculus}
The rho-calculus is a small and simple calculus which is We are using rho-calculus because it is a small and simple calculus which is fast to learn.
It is developed and described by L. G. Meredith and M. Radestock in \citep{Meredith2005}.
We will use rho-calculus to model a system, and later a security type rule will be applied.
Because rho-calculus is a small calculus, there will not be many reduction rules to look through, and the type rules for the reduction rules will then be faster to make than with a bigger calculus.
The rho-calculus have a syntax for parallelism which will be useful for modelling client/server systems, were security can be an issue, and security can be a major part of the specification. 

\begin{align*}
    P  ::= \; &  \nil & \text{nil} \\
      & \mid \inp{x}{y}P & \text{input} \\
      & \mid \lift{x}{M} & \text{lift} \\
      & \mid \drop{x} & \text{drop} \\
      & \mid P \para Q & \text{parallel} \\[3mm]
    x,y ::= \; & \quot{P} & \text{quote}\\
\end{align*}

The syntax of the rho-calculus is described as follows.

\subsection{Processes}
\begin{description}
\item[Nil] Does nothing.
\item[Input] Receives a term lifted on channel \textit{x} and proceeds as process \textit{P} with the term replacing \textit{y}.
\item[Lift] Lifts a term \textit{M} on channel \textit{x}.
\item[Drop] Drops a name \textit{x}, so it becomes a process, and proceeds as the resulting process.
\item[Parallel] Proceeds as both process \textit{P} and process \textit{Q} in parallel.
\end{description}


\subsection{Names}
\begin{description}
\item[Quote] Quotes a process \textit{P}, so it becomes a name.
\end{description}

%\subsubsection{Boolean Arithmetic}



\subsection{Additional Syntax}
Although it is possible to implement a blockchain protocol in the rho-calculus, it would require convoluted structures to express even simple concepts.
We choose to extend the rho-calculus to allow easier modeling of those concepts.\\
\\
We add an additional syntax rule for terms, where $M$ ranges over terms, $n$ ranges over numbers, $s$ ranges over strings and $f$ ranges over operations.
\begin{align*}
M::=\; & n \mid s \mid (M_1,...,M_k) \mid fM &for\ all\ k \geq 2
\end{align*}
In addition to terms and processes we need to use conditional processes. This requires boolean expressions, denoted by \ensuremath{\phi}.
\begin{align*}
P::=& \ [\phi] P\\
\phi ::=& \ M_1\gamma M_2|\phi\land\phi|\phi\lor\phi|\neg\phi|\top|\bot\\
\gamma ::=& \ =|\neq|<|>|\leq|\geq
\end{align*}
To ease the writing of conditional branching, we allow an if-then-else syntax.
\begin{align*}
	\If{\phi}{P}{Q} = [\phi].P | [\neg \phi].Q
\end{align*}
\subsection{Replication}
Our implementation makes use of replication, denoted by $!P$, to express inexhaustible processes.
The replication $!P$ proceeds as an arbitrary number of copies of $P$ in parallel. This is shown by the reduction example in \figref{fig:reductionexample}. The replication was developed by L. G. Meredith and M. Radestock\citep{Meredith2005}.
%\begin{figure}[h]
%    \begin{center}
%        \begin{tabular}[c]{cll}
%            & !P & \runa{Initial} \\

%            $\equiv$ & \lift{x}{\inp{x}{y}(\lift{x}{\drop{y}}\para\drop{y})\para P}\para\inp{x}{y}(\lift{x}{\drop{y}}\para \drop{y}) & \runa{Substitution} \\

%            $\ra$ & \lift{x}{\drop{\quot{\inp{x}{y}(\lift{x}{\drop{y}}\para\drop{y})\para P}}}\para\drop{\quot{\inp{x}{y}(\lift{x}{\drop{y}}\para\drop{y})\para P}} & \runa{Communication} \\

%            $\equiv$ & \lift{x}{\inp{x}{y}(\lift{x}{\drop{y}}\para\drop{y})\para P}\para\inp{x}{y}(\lift{x}{\drop{y}}\para \drop{y})\para P & \runa{DropQuote} \\

%            $\equiv$ & !P\para P & \runa{Substitution}
%        \end{tabular}
%    \end{center}
%    \caption{Reduction example of replication}
%    \label{fig:reductionexample}
%\end{figure}


\begin{figure}[h]
    \begin{align}
        &!P \tag{Initial} \\
        &\equiv \lift{x}{\inp{x}{y}(\lift{x}{\drop{y}}\para\drop{y})\para P}\para\inp{x}{y}(\lift{x}{\drop{y}}\para \drop{y}) \tag{Substitution} \\
        &\ra \lift{x}{\drop{\quot{\inp{x}{y}(\lift{x}{\drop{y}}\para\drop{y})\para P}}}\para\drop{\quot{\inp{x}{y}(\lift{x}{\drop{y}}\para\drop{y})\para P}} \tag{Comm}\\
        &\equiv \lift{x}{\inp{x}{y}(\lift{x}{\drop{y}}\para\drop{y})\para P}\para\inp{x}{y}(\lift{x}{\drop{y}}\para \drop{y})\para P \tag{DropQuote} \\
        &\equiv\ 
        !P\para P \tag{Substitution}
    \end{align}
    \caption{Reduction example of replication}
    \label{fig:reductionexample}
\end{figure}

\FloatBarrier
