\section{Introduction}
The distributed algorithm blockchain has become very popular these years. The algorithm became widely known from the Bitcoin protocol, where it was used to create a decentralized cryptocurrency\citep{website:blockchain}. The blockchain consists of a chain of blocks, hence the name, where anyone in the network can add new blocks to the chain. To ensure that the blocks are valid consensus in the network has to be achieved. This is done with a consensus algorithm. Depending on the algorithm the process is different but they all have one property in common and that is that over 50\% of the network has to agree and thereby give consensus of the validity of the new block, for it to be added to the chain.\\
Bitcoin is not the only implementation of the blockchain algorithm. One such example is the RChain protocol. The RChain is a blockchain algorithm currently in the descriptive phase designed to be written in the rho-lang, which is an concurrency-oriented programming language. The language is designed to easily translate rho-calculus to rho-lang\citep{website:rho-lang}. Although in this article RChain is mainly used as an inspiration, and also one of the reasons for using rho-calculus. Our implementation written in the rho-calculus is based on the naivechain, which is an example of a blockchain implementation\citep{naivechain}.\\

The rho-calculus is a very expressive calculus but for our own convenience, we choose to extend the calculus that can be read in section \ref{ch:rho-calculus}. The extensions where added to simplify our blockchain implementation as this article does not focus on the rho-calculus but merely uses it, but the implementation can be designed without our extensions.\\

Since the algorithm is often used in security essential systems, then verifying that the security aspects are uphold is very relevant. One such aspect is the integrity of the chain.
The Integrity is an important aspect of the blockchain protocol, as one concept of the blockchain algorithm is that the modification to the chain is the addition of new blocks. Therefor it is important making sure that the protocol follows that concept. In this article it will be done with a type-system. The system will consists of types representing security levels, that is fully described in section \ref{ch:type-security}.
