\section{Introduction}
The distributed algorithm blockchain has become very popular these years. The algorithm became widely known because of Bitcoin, to create a decentralized cryptocurrency.\citep{website:blockchain} The popularity has made others make their own blockchain implementation, one example of such is the RChain protocol. The RChain protocol is at the time of writing only a description, as to how it could be implemented in the calculus rho-calculus. We took inspiration from the RChain protocol, which is also one of the reasons we chose to use the rho-calculus for our implementation. Further reasons will be discussed later in the article. Our implementation is based upon the implementation of a blockchain called naivechain\cite{naivechain}.

\subsection*{Blockchain introduction}
The blockchain is a distributed algorithm (protocol) mostly known from bitcoin. The protocol consists of a chain of blocks, hence the name, where anyone in the network can add new blocks to the chain. To ensure that the blocks are valid consensus has to be achieved. This is done with a consensus protocol and depending on the protocol the process is different, but a common property is that the network will eventually agree on the same chain.\\
Since the blockchain is often used in security critical systems, the protocol needs to have a verified security system.\cite{website:integrity} In this article, we will look upon how we could ensure the integrity in the blockchain protocol. This is done by developing a type rule in rho-calculus which will ensure the integrity of the system.
