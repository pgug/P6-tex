\section{Conclusion}
We have shown that the rho-calculus is a usable calculus, we managed to implement a blockchain protocol, and made a type system which ensures the integrity of a system. By showing that the rho-calculus is useful can be more used in the future, rho-calculus have been out since 2005, but have not been seen used in many occasions. The RChain project could very well have some potential, and we have shown that it is possible to implement a blockchain protocol in the rho-calculus, so the RChain should also be possible to make.\\

For further work we could implement other security feature in the rho-calculus such that instead of only proving the integrity of a system, it could hold its authenticity or its immutability, other security features could also be implemented. The rho-calculus could also be implemented into a type checker or model checker, and in that way check a model for its integrity or for other security features. But with the work we have done on the rho-calculus we have shown that it is useful and can be use in other occasions, because it is as strong and expressive as the pi calculus as shown in \citep{Meredith2005}, but we have shown that we could implement a blockchain protocol in it.


\subsection*{Acknowledgements}
We would like to thank Hans Hüttel our supervisor for helping us with this project, for our discussion on the calculus and type system and for guiding us in the right direction.
