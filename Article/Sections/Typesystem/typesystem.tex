\section{Type Security}
In order to determine whether a protocol is secure or not, we need a model that can express security.
We will construct a type system on top of our existing calculus, that will include security levels.
That way we can define integrity as a behaviour of our types.
\begin{align*}
	T::=&\;l \tag{Security Level}\\
	&\mid ch[T] \tag{Channel}\\
	&\mid T_1 \times...\times T_n \tag{Tuple}\\
	&\mid B^l \tag{Basic}\\
	B::=&\;int 
	\mid string 
	\mid list 
	\mid bool 
\end{align*}

\noindent The syntax of our type system is shown above. It is described as follows:

\begin{description}
\item[Security levels] range over the natural numbers $\mathbb{N}_0$. This ensures an ordering between different security levels, where $0$ is considered the lowest level of security.
\item[Channels] allow terms of type $T$ to be sent and received.
\item[Tuples] have types composed of all the types of the contained terms.
\item[Basic Types] have a given security level $l$.
\end{description}

\noindent A type judgment $x:T$ denotes that $x$ is well-typed at type $T$.
These judgments can be collected to form a type environment, denoted $E$, that partially maps names to types.
\begin{align*}
E: names \rightharpoonup types
\end{align*}

\noindent $E\vdash x:T$ denotes that $x$ is well-typed at type $T$ in context $E$.
If $x$ is not given a type, then it is simply well-typed. $E(x)$ denotes the actual type of $x$, while $E\vdash x:T$ 

\begin{align*}
\tag{Lookup}\label{rule:lookup} &\infrule{E(x)=T}{E\vdash x:T}
\end{align*}

The \ref{rule:lookup} rule states that, given context $E$, if the actual type of $x$ is $T$, then $x$ is also well-typed at type $T$. 
In order to rule if $x$ is well-typed at other types, we need to introduce subtyping, denoted by the relation $\leq$. The ordering of the security levels apply to the subtype relation, so that $l$ is a subtype of $l'$ if $l\leq l'$

\begin{align*}
\tag{LessOrEqual}\label{rule:lessorequal} &\infrule{T=T'}{T\leq T'}
\end{align*}

The \ref{rule:lessorequal} rule states that equal types are also subtypes of each other.

\begin{align*}
\tag{Transitive}\label{rule:transitive} &\infrule{T\leq T''\quad T''\leq T'}{T\leq T'}
\end{align*}

The \ref{rule:transitive} rule gives subtyping a transitive property.

\begin{align*}
\tag{Subsumption}\label{rule:subsumption} &\infrule{E\vdash x:T\quad T\leq T'}{E\vdash x:T'}
\end{align*}

The \ref{rule:subsumption} rule states that, given context $E$, if $x$ is well-typed at $T$ and $T$ is a subtype of $T'$, then $x$ is also well-typed at type $T'$. This is what allows a subtyping to maintain well-typing.

\begin{align*}
\tag{BasicSubBasic} &\infrule{l\leq l'}{B^l\leq B^{l'}}\\
\tag{TupleSubTuple} &\infrule{T_i\leq T'}{T_1\times...\times T_i\times...\times T_n\leq T_1\times...\times T'\times...\times T_n}\\
\tag{Level-level} &\level{l}=l\\
\tag{Channel-level} &\level{ch[T]} = \level{T}\\
\tag{Tuple-level} &\level{T_1\times...\times T_n}=\max_{i=1}^{n}{\level{T_i}}\\
\tag{Basic-level} &\level{B^l}=l\\
\tag{TypeSubLevel} &T\leq \level{T}\\
\tag{TermAsChannel} &\infrule{E(x)=T\quad \level{T}=l}{E\vdash x:ch[l]}
\end{align*}

\begin{align*}
\tag{Nil} &E \vdash \nil\\
\tag{Input} &\infrule{E\vdash x:ch[T]\quad E,y:T\vdash P}{E \vdash \inp{x}{y}P}\\
\tag{Lift} &\infrule{E\vdash x:ch[T]\quad E \vdash M:T}{E\vdash \lift{x}{M}}\\
\tag{Drop} &\infrule{E \vdash P}{E\vdash \drop{x}}& where \drop{x} \equiv P\\
\tag{Parallel} &\infrule{E\vdash P \quad E \vdash Q}{E \vdash P \para Q}\\
\tag{Quote} &\infrule{E \vdash x}{E\vdash \quot{P}}& where\ \quot{P} \equiv_N x
\end{align*}

\begin{align*}
\tag{Condition} &\infrule{E\vdash \phi:bool\quad E\vdash P}{E \vdash [\phi] P}\\
\tag{Relation} &\infrule{E\vdash M_1:T \quad E\vdash M_2:T}{E\vdash M_1 \gamma\ M_2:bool}\\
\tag{And} &\infrule{E\vdash \phi _1:bool \quad E\vdash \phi _2:bool}{E\vdash \phi _1 \land \phi _2:bool}\\
\tag{Or} &\infrule{E\vdash \phi _1:bool \quad E\vdash \phi _2:bool}{E\vdash \phi _1 \lor \phi _2:bool}\\
\tag{Negation} &\infrule{E\vdash \phi :bool }{E\vdash \neg \phi :bool}\\
\tag{Operation} &\infrule{E\vdash f:T\ra T'\quad E\vdash M:T}{E\vdash fM:T'}\\
\tag{Tuple} &\infrule{E\vdash M_i:T_i \quad (1\leq i\leq n)}{E\vdash (M_1,...,M_n):T_1\times...\times T_n}\\
\tag{Int} &E\vdash n: int^l\\
\tag{String} &E\vdash s:string^l\\
\end{align*}

%\textit{max(T)} is a operation which returns the maximum value off its input \textit{T}, according to the input \textit{T}s order.

%\begin{align*}
%\tag{Append} &\infrule{E \vdash x:T^l \quad E \vdash y:list^l}{E \vdash append(x,y):list^l}\\
%\tag{Head} &\infrule{E \vdash x:list^l}{E \vdash head(x):l}\\
%\tag{Tail} &\infrule{E \vdash x:list^l}{E \vdash tail(x):list^l}\\
%\tag{newBlock}& \infrule{E \vdash data:T \quad E \vdash previus : T}{E \vdash newBlock(data, previus) :T}\\
%\tag{getLatest}& \infrule{E \vdash chain :T}{E \vdash getLatest(chain):T}\\
%\tag{addToChain}& \infrule{E \vdash block:T \quad E \vdash chain:T}{E \vdash addToChain(block, chain):T}\\
%\tag{getLenght}& \infrule{E \vdash chain:T}{E \vdash getLenght(chain):int}\\
%\tag{isValid}& \infrule{E \vdash chain:T}{E \vdash isValid(chain):bool}
%\end{align*}

\begin{align*}
\tag{Action} & \infrule{E \vdash P \quad P\ra P'}{E\vdash P'}\\
\tag{Not Wrong}& \infrule{E \vdash P}{P \nrightarrow wrong}\\
\tag{Para Wrong}& \infrule{P \ra wrong}{P\para Q \ra wrong}\\
\tag{Wrong}& \infrule{E(x) = ch^l[T] \quad E \vdash M:T \quad L(T) > l }{\lift{x}{M} \ra wrong}
\end{align*}


%\fig{M: secret,c:secure\vdash \lift{c}{M}}{If \textit{M} is of type \textit{secret}, then \textit{c} should be of type \textit{secure}}{fig:sec}
%
%\fig{\frac{\Gamma \vdash M:Secret \quad \Gamma \vdash c:Secure}{\Gamma \vdash \lift{c}{M}}}{}{secret}
%
%\fig{\frac{\Gamma \vdash M:Public \quad \Gamma \vdash c:Insecure}{\Gamma \vdash \lift{c}{M}}}{}{public}

\FloatBarrier
