\section{Type System}\label{ch:type-security}
In order to determine whether a protocol is secure or not, we need a model that can express security.
We will construct a type system on top of our existing calculus, that will include security levels.
These levels allow us to isolate the client, by giving it higher security levels than the environment surrounding it.
If the type system prevents anything with some security level from being accessible to an environment with a lower security level, then the client gains integrity even in a malicious environment.
Because these levels are part of our type system, we can define integrity as a behaviour of our types.

\begin{align*}
    T::=&\;U^l\\    	
    U::=&\;B \tag{Basic Type}\\
        &\mid ch[T] \tag{Channel}\\
        &\mid T_1 \times...\times T_k \tag{Tuple}\\
        &\mid T\ra T \tag{Operation}\\
    B::=&\;number 
        \mid string
        \mid bool\\
\end{align*}
The syntax of our type system is shown above. It is described as follows:

\begin{description}
    \item[Channels] allow terms of type $T$ to be sent and received.
    \item[Tuples] have types composed of all the types of the contained terms.
    \item[Operations] have an argument type and a resulting type.
    \item[Basic Type] are types that do not consist of other types.
\end{description}

The security level $l$ ranges over the natural numbers $\mathbb{N}_0$.
This ensures an ordering between different security levels, where $0$ is considered the lowest level of security.
In the rest of this section, we denote the security level of type $T$ as $\level{T}$.
This is defined by the following:

\begin{align*}
    \tag{Level}\level{U^l}\eqdef l
\end{align*}

\subsection{Type Validation}
With the syntax of our type system, it is possible to make types with some security level that consist of types with lower security level.
This could allow anything with some security level to disguise itself as a lower security level, allowing it to be accessed by environments of lower security level.
To avoid this we introduce type validation, where $\valid{T}$ denotes that the type $T$ is valid.

\begin{align*}
\tag{Subtype}\label{rule:Subtype} &\infrule{l\leq l'}{U^l\leq U^{l'}}
\end{align*}
The \ref{rule:Subtype} rule states that if $l$ is at most $l'$, then a type with security level $l$ is a subtype of a type with security level $l'$ that shares the same specialization type.
For example, $string^2$ is a subtype of $string^4$, since $2\leq4$ and the specialization types of both types is $string$.

\subsection{Type Validation}
With the syntax of our type system, it is possible to make types with some security level that allow access to something with a higher security level.
This allows environments of lower security level to access something with a higher security level, which should not be allowed.
To avoid this we introduce type validation, where $\valid{T}$ denotes that the type $T$ is valid, meaning that $T$ does not give access to types with a higher security level than the security level of $T$.

\begin{align*}
    \tag{ValidBasicTerm}\label{rule:ValidBasicTerm} &\valid{B^l}
\end{align*}
The \ref{rule:ValidBasicTerm} rule state that the basic term types are trivially valid.
This is true since they do not consist of any other types and therefore can not give access to higher security levels.

\begin{align*}
    \tag{ValidOperation}\label{rule:ValidOperation} &\infrule{\valid{U^l}\quad l\leq l'}{\valid{(T\ra U^l)^{l'}}}
\end{align*}
The \ref{rule:ValidOperation} rule states that an operation type is valid if its resulting type is valid and the operation type has at least the security level of its resulting type.
This is because the security level of the resulting type determines what security level the operation allows access to.
The type system ensures that any argument given to an operation is a valid type, so it would be redundant to validate the argument type of the operation type here, since it would validate the same type multiple times.

\begin{align*}
    \tag{ValidChannel}\label{rule:ValidChannel} &\infrule{\valid{U^l}\quad l\leq l'}{\valid{ch[U^l]^{l'}}}
\end{align*}
The \ref{rule:ValidChannel} rule states that a channel type is valid if the type communicated by the channel type is valid and has at most the security level of the channel type.
This ensures that anything of some security level can not be communicated on a channel of lower security and possibly making it accessible to an environment of lower security level.

\begin{align*}
    \tag{ValidTuple}\label{rule:ValidTuple} &\infrule{\forall i\ \valid{T_i}\quad \forall i\ \level{T_i}\leq l}{\valid{(T_1 \times...\times T_k)^l}}& where\ 1\leq i\leq k
\end{align*}
The \ref{rule:ValidTuple} rule states that a tuple type is valid if all its composite types are valid and all their security levels are at most the security level of the tuple type.
This ensures that a tuple with some security level does not contain terms with a higher security level, since that could allow environments with some security level to access something of a higher security level.\\

Note that all the validation rules ensure that if the security level of a valid type were increased, the type would still be valid. Together with the \ref{rule:Subtype} rule we can conclude that a type is valid if it has a valid subtype.

\subsection{Well-typing}
In order to give types to processes in our extended rho-calculus, we need to introduce well-typing.
A type judgment $x:T$ denotes that $x$ is well-typed at type $T$.
These judgments can be collected to form a type environment, denoted $E$, that partially maps $\mathbf{Names}$ to $\mathbf{Types}$.

\begin{align*}
    E: \mathbf{Names} \rightharpoonup \mathbf{Types}
\end{align*}
$E\vdash x:T$ denotes that $x$ is well-typed at type $T$ in context $E$.
$E\vdash x$, where no type is provided, denotes that $x$ is simply well-typed.
$E(x)$ denotes the actual type of $x$.

\begin{align*}
    \tag{Lookup}\label{rule:lookup} &\infrule{E(x)=T\quad\valid{T}}{E\vdash x:T}
\end{align*}

The \ref{rule:lookup} rule states that, given context $E$, if the actual type of $x$ is $S$ and $S$ is a valid type, then $x$ is well-typed at type $S$.
The validation of $S$ ensures that well-typed processes can not have invalid types originating from the type environment.

\subsection{Subtyping}
In order to state that an process is well-typed at other types, we need to introduce subtyping.
The subtyping $S\leq S'$ denotes that $S$ is a subtype of $S'$. 

\begin{align*}
    \tag{Subtype}\label{rule:subtype} &\infrule{l\leq l'}{U^l\leq U^{l'}}
\end{align*}

The \ref{rule:subtype} rule states that if $l$ is at most $l'$, then a term with security level $l$ is a subtype of a term with security level $l'$ that shares the same specialization type.
For example, $string^2$ is a subtype of $string^4$, since $2\leq4$ and the specialization types of both terms is $string$.

\begin{align*}
    \tag{Subsumption}\label{rule:subsumption} &\infrule{E\vdash x:T\quad T\leq T'}{E\vdash x:T'}
\end{align*}
The \ref{rule:subsumption} rule states that, given context $E$, if $x$ is well-typed at $T$ and $T$ is a subtype of $T'$, then $x$ is also well-typed at type $T'$.
Together with the \ref{rule:Subtype} rule this states that $x$ remains well-typed at a type that has a higher security level.

\subsection{Well-typing of processes}
In order to use the type system with processes in our extended rho-calculus, we need to define the well-typing of all processes.
This is done by providing type rules that cover our syntax and consequently any processes in our syntax.
\begin{align*}
    \tag{Nil}\label{rule:Nil} &E \vdash \nil
\end{align*}
The \ref{rule:Nil} rule states that any nil process is trivially well-typed.
This is true since the process contains no types.

\begin{align*}
    \tag{Input}\label{rule:Input} &\infrule{E\vdash x:ch[T]^l\quad E,y:T\vdash P}{E \vdash \inp{x}{y}P}
\end{align*}
The \ref{rule:Input} rule states that, given context $E$, if $x$ is well-typed at a channel type, that communicates type $T$, and $P$ is well-typed in a context where $y$ is well-typed at $T$, then $x(y).P$ is well-typed.
This ensures that communication is done on a channel and that $y$ is well-typed at the type communicated by the channel when proceeding with $P$.

\begin{align*}
    \tag{Lift}\label{rule:Lift} &\infrule{E\vdash x:ch[T]^l\quad E \vdash M:T}{E\vdash \lift{x}{M}}
\end{align*}
The \ref{rule:Lift} rule states that, given context $E$, if $x$ is well-typed at a channel type, that communicates type $T$ and $M$ is well-typed at $T$, then $\lift{x}{M}$ is well-typed.
This ensures that communication is done on a channel and that $M$ is well-typed at the type communicated by the channel.

\begin{align*}
    \tag{Drop}\label{rule:Drop} &\infrule{E \vdash P}{E\vdash \drop{x}}& where \drop{x} \equiv P
\end{align*}
The \ref{rule:Drop} rule states that, given context $E$, if the resulting process of dropping $x$ is well-typed, then dropping $x$ is well-typed.

\begin{align*}
    \tag{Parallel}\label{rule:Parallel} &\infrule{E\vdash P \quad E \vdash Q}{E \vdash P \para Q}
\end{align*}
The \ref{rule:Parallel} rule states that, given context $E$, two well-typed processes in parallel form a well-typed process.

\begin{align*}
\tag{Condition}\label{rule:Condition} &\infrule{E\vdash \phi:bool^l\quad E\vdash P}{E \vdash [\phi] P}
\end{align*}
The \ref{rule:Condition} rule states that, given context $E$, if $\phi$ is well-typed at $bool$ and $P$ is well-typed, then the conditional process $[\phi]P$ is also well-typed. This ensures that the condition always evaluates to a boolean value.

\begin{align*}
    \tag{Quote}\label{rule:Quote} &\infrule{E \vdash x:T}{E\vdash \quot{P}:T}& where\ \quot{P} \equiv_N x
\end{align*}
The \ref{rule:Quote} rule states that, given context $E$, if the resulting name of quoting $P$ is well-typed at type $T$, then quoting $P$ is also well-typed at $T$.

\begin{align*}
\tag{Tuple}\label{rule:Tuple} &\infrule{\forall i\ E\vdash M_i:T_i\quad \forall i\ \level{T_i}\leq l}{E\vdash (M_1,...,M_k):(T_1\times...\times T_k)^l}& where\ 1\leq i\leq k
\end{align*}
The \ref{rule:Tuple} rule states that, given context $E$, if some terms are well-typed at some types, and the highest security level of those types is $l$, then a tuple containing those terms are well-typed at the composition of the types the terms are well-typed at where the composition has a security level of $l$.
This ensures that the tuple type can not be composed of types with higher security level than itself.

\begin{align*}
\tag{Operation}\label{rule:Operation} &\infrule{E\vdash f:T\ra T'\quad E\vdash M:T}{E\vdash fM:T'}
\end{align*}
The \ref{rule:Operation} rule states that, given context $E$, if $f$ is well-typed at a operation type and $M$ is well-typed at the argument type of that operation type, then $fM$ is well-typed at the resulting type of that operation type.
This ensures that the argument type and the resulting type are coherent with the types defined by the operation type.

\begin{align*}
\tag{Relation}\label{rule:Relation} &\infrule{E\vdash M_1:T\quad E\vdash M_2:T\quad\level{T}=l}{E\vdash M_1 \gamma\ M_2:bool^{l}}
\end{align*}
The \ref{rule:Relation} rule states that, given context $E$, if the terms $M_1$ and $M_2$ are well-type at type $T$ with security level $l$, then $M_1\gamma M_2$ is well-typed at $bool^l$. This ensures that the two terms are comparable with each other and that the security level of the resulting type is at least the security level of the argument types.

\begin{align*}
    \tag{And}\label{rule:And} &\infrule{E\vdash \phi_1:bool^l\quad E\vdash \phi_2:bool^l}{E\vdash \phi _1 \land \phi _2:bool^{l}}\\
    \tag{Or}\label{rule:Or} &\infrule{E\vdash \phi_1:bool^l\quad E\vdash \phi_2:bool^l}{E\vdash \phi _1 \lor \phi _2:bool^{l}}\\
    \tag{Negation}\label{rule:Negation} &\infrule{E\vdash \phi :bool^l }{E\vdash \neg \phi :bool^l}
\end{align*}

The \ref{rule:And}, \ref{rule:Or} and \ref{rule:Negation} rules states that, given context $E$, boolean operators are well-typed if both the arguments and the result are well-typed at $bool^l$. This ensures that the security level of the resulting type is at least the security level of the argument types.

\begin{align*}
\tag{True}\label{rule:True} &E\vdash\top:bool^0\\
\tag{False}\label{rule:False} &E\vdash\bot:bool^0\\
\tag{Number}\label{rule:Number} &E\vdash n:{number^0}\\
\tag{String}\label{rule:String} &E\vdash s:{string^0}
\end{align*}
The \ref{rule:True} and \ref{rule:False} rules state that $\top$ and $\bot$ are trivially well-typed at $bool^0$.
The \ref{rule:Number} rule states that $n$ is trivially well-typed at $number^0$.
The \ref{rule:String} rule states that $s$ is trivially well-typed at $string^0$.
These rules ensure that any values used statically in the protocol have the lowest level of security possible, since the protocol is considered accessible to all environments.

\section{Type Security}

\begin{theorem}[Wrong]
\ensuremath{E \vdash P \ra wrong} \\
\textbf{Definition: Wrong}
To show when \ensuremath{P} goes \ensuremath{wrong} we need show every case where \ensuremath{P} goes \ensuremath{wrong}.\\
\textbf{Lift 1} \begin{align*}
\frac{E \nvdash M}{E \vdash \lift{x}{M} \ra wrong}
\end{align*}

\textbf{Lift 2} \begin{align*}
\frac{E \vdash M:T_1 \quad E(x)=ch[T_2]}{E \vdash \lift{x}{M} \ra wrong} &\quad where T_1 \neq T_2
\end{align*}

\textbf{Input} \begin{align*}
\frac{\exists M,\quad E \vdash P\{ M/ y\} \ra wrong \quad E \vdash M:T \quad E(x)= ch[T]} {E \vdash \inp{x}{y}P \ra wrong}\\
\end{align*}

\textbf{Parallel} \begin{align*}
\frac{E \vdash P \ra wrong}{E \vdash P \para Q \ra wrong}
\end{align*}

\textbf{Structural Congruens} \begin{align*}
\frac{E \vdash Q \quad P \equiv Q}{E \vdash P \ra wrong}
\end{align*}

\textbf{Drop}\begin{align*}
\frac{\drop{x} \equiv P \quad E \vdash P \ra wrong}{E \vdash \drop{x} \ra wrong}
\end{align*}

\textbf{Quot}\begin{align*}
\frac{\quot{P} \equiv _N x \quad E \vdash P \ra wrong}{E \vdash \drop{x} \ra wrong}
\end{align*}

\textbf{Condition}\begin{align*}
\frac{E \vdash P \ra wrong}{E \vdash [\phi]P \ra wrong}
\end{align*}


\end{theorem}

\begin{theorem}[Subject Reduction]
If \ensuremath{E \vdash P} and \ensuremath{P\ra P'} then \ensuremath{E \vdash P'}.
To show that, we have to prove that \ensuremath{P \ra P'} holds for the rules, parallel, communication and structural congruent. We can do that by prove it by induction for the cases \ensuremath{n} and \ensuremath{n+1}.

\begin{proof}[Parallel]
\begin{align*}
    \frac{P \ra P' \quad =n}{P \para Q \ra P' \para Q} =n+1 \tag{Parallel}
\end{align*}
If \ensuremath{E \vdash P} and \ensuremath{P \ra P'} then \ensuremath{E \vdash P'} We want to show that if \ensuremath{E \vdash P \para Q} then \ensuremath{E \vdash P' \para Q}. The type rule for parallel say \ensuremath{\frac{E \vdash P\quad E \vdash Q}{E \vdash P \para Q}} so if \ensuremath{E \vdash P'} then \ensuremath{\frac{E \vdash P' \quad E \vdash Q}{E \vdash P' \para Q}}
\end{proof}

\begin{proof}[Communication]
\begin{align*}
    \infrule{}{\lift{x}{M}\para\inp{x}{y}P\ra P\{M/y\}} \tag{Communication}
\end{align*}

\begin{lemma}[Substitution]
If \ensuremath{E,x:T \vdash P} and \ensuremath{E_2 \vdash M:T} then \ensuremath{E, E_2 \vdash P\{M/x \}}
\end{lemma}

\end{proof}

\begin{proof}[Structural Congruent]
\begin{align*}
    \infrule{P \equiv P' \quad P' \ra Q' \quad Q' \equiv Q}{P \ra Q} \tag{Structural Congruent}
\end{align*}

If \ensuremath{E \vdash P} then \ensuremath{E \vdash P'} We will want to show that if \ensuremath{E \vdash Q} then \ensuremath{E \vdash Q'}

\begin{lemma}[Equivalent]
If \ensuremath{P \equiv Q} and \ensuremath{E \vdash P} then \ensuremath{E \vdash Q}
\end{lemma}

By applying lemma equivalent to the structural congruent, we see that because \ensuremath{P \equiv Q} and \ensuremath{E \vdash P} then \ensuremath{E \vdash Q} and because \ensuremath{P \ra P'} and \ensuremath{E \vdash P'} then \ensuremath{E \vdash Q'}.
\end{proof}

\end{theorem}


\begin{theorem}
If \ensuremath{E \vdash P} then \ensuremath{E \vdash P \nrightarrow wrong}\\
We need to show by induction that every reduction holds by their assumption and do not goes to \ensuremath{wrong}.
\begin{proof}[Parallel]
\begin{align*}
\frac{E \vdash P \quad E \vdash Q }{E \vdash P \para Q}
\end{align*}

By induction hypothesis we have that: 
\begin{align*}
E \vdash P \nrightarrow wrong\\
E \vdash Q \nrightarrow wrong
\end{align*}
That is way we can say that:
\begin{align*}
\frac{}{E \vdash P \para Q \nrightarrow wrong}
\end{align*}

Because the wrong predicate for parallel says:
\begin{align*}
\frac{E \vdash P \ra wrong}{E \vdash P \para Q \ra wrong}
\end{align*}
And we can prove by structural equivalences that the same holds for \ensuremath{Q}
\end{proof}

\begin{proof}[Nil]
Because there is not a wrong predicate for nil, nil can not go wrong, and we do not have to show that nil will not go wrong.
\end{proof}

\begin{proof}[Lift]
\begin{align*}
\frac{E \vdash M:T_1 \quad E \vdash x:ch[T_2] \quad T_1 = T_2}{E \vdash \lift{x}{M}}
\end{align*}
By induction hypothesis we have that:
\begin{align*}
E \vdash M:T_1\\
E \vdash x:ch[T_2]\\
T_1 = T_2
\end{align*}
That is why we can say that:
\begin{align*}
\frac{}{E \vdash \lift{x}{M} \nrightarrow wrong}
\end{align*}

Because the wrong predicate for Lift says:
\begin{align*}
\frac{E \nvdash M}{E \vdash \lift{x}{M} \ra wrong}\\
\frac{E \vdash M:T_1 \quad E \vdash x:ch[T_2] \quad T_1 \neq T_2}{E \vdash \lift{x}{M} \ra wrong}
\end{align*}

\end{proof}

\begin{proof}[Input]
\begin{align*}
\frac{E \vdash x:ch[T] \quad E,y:t \vdash P}{E \vdash \inp{x}{y}P}
\end{align*}
By induction hypothesis we have that:
\begin{align*}
E \vdash x:ch[T]\\
E \vdash y:T \vdash P \nrightarrow wrong
\end{align*}

That is way we can say that:
\begin{align*}
\frac{}{E \vdash \inp{x}{y}P \nrightarrow wrong}
\end{align*}

Because the wrong predicate for input says:
\begin{align*}
\frac{\exists M, \quad E \vdash P\{M/y\}\ra wrong \quad E\vdash M\vdash T \quad E(x)=ch[T]}{E \vdash \inp{x}{y}P \ra wrong}
\end{align*}
\end{proof}

\begin{proof}[Drop]
\begin{align*}
\frac{E \vdash P}{E \vdash \drop{x}}
\end{align*}

By induction hypothesis we have that:
\begin{align*}
E \vdash P \nrightarrow wrong
\end{align*}

That is way we can say that:
\begin{align*}
\frac{}{E \vdash \drop{x} \nrightarrow wrong}
\end{align*}

Because the wrong predicate for drop says:
\begin{align*}
\frac{\drop{x} \equiv P \quad E \vdash P \ra wrong}{E \vdash \drop{x} \ra wrong}
\end{align*}
\end{proof}

\begin{proof}[Quote]
\begin{align*}
\frac{E \vdash x:T \quad \quot{P} \equiv _N x}{E \vdash \quot{P}:T}
\end{align*}

By induction hypothesis we have that:
\begin{align*}
E \vdash \drop{x} \nrightarrow wrong
\end{align*}

That is way we can say that:
\begin{align*}
\frac{}{E \vdash \drop{x} \nrightarrow wrong}
\end{align*}

Because the wrong predicate for quote is:
\begin{align*}
\frac{\quot{P} \equiv _N x \quad E \vdash P ra wrong}{E \vdash \drop{x} \ra wrong}
\end{align*}

\end{proof}

\begin{proof}[Condition]
\begin{align*}
\frac{E \vdash \phi : bool \quad E \vdash P}{E \vdash [\phi]P}
\end{align*}

By induction hypothesis we have that:
\begin{align*}
E \vdash P \nrightarrow wrong
\end{align*}

That is way we can say that:
\begin{align*}
\frac{}{E \vdash [\phi ]P \nrightarrow wrong}
\end{align*}

Because the wrong predicate for condition is:
\begin{align*}
\frac{E \vdash P \ra wrong}{E \vdash [\phi ]P \ra wrong}
\end{align*}
\end{proof}
\end{theorem}




%\begin{align*}
%    \tag{Wrong}& \infrule{E(x) = ch[T_1] \quad E \vdash M:T_2 \quad \level{T_2} > \level{T_1} }{\lift{x}{M} \ra wrong}
%\end{align*}


\FloatBarrier
