\section{Type Security}
In this section we will show that any process, that is well-typed in our type system, can never break the integrity of a protocol.
\subsection{Wrong-predicate}
To ensure that a well-typed process can never break the integrity of a protocol, we first need to define when a process breaks the integrity of a protocol.
The wrong-predicate, $E \vdash P \ra wrong$, denotes that the well-typed process $P$ can go $wrong$ and break the integrity of a protocol. In order to define when $P$ goes $wrong$, we will present every case of this.

    \begin{align*}
        \tag{Lift 1}\frac{E \nvdash M}{E \vdash \lift{x}{M} \ra wrong}
    \end{align*}\\
    If $M$ is not well typed under $E$ then the lifting will go $wrong$.\\
        
    \begin{align*}
            \tag{Lift 2}\frac{E \vdash M:T_1 \quad E(x)=ch[T_2]^l}{E \vdash \lift{x}{M} \ra wrong} &\quad \text{where}\ T_1 \neq T_2
    \end{align*}
    If $M$ is of type $T_1$ and $x$ is of type channel with type $T_2$, and $T_1$ and $T_2$ are not equal then the lifting will go wrong. We may not lift an $M$ on a channel which is not of the same type.\\

    \begin{align*}
        \tag{Input}\frac{E \vdash P\{ M/ y\} \ra wrong \quad E \vdash M:T \quad E(x)= ch[T]^l} {E \vdash \inp{x}{y}P \ra wrong}\\
    \end{align*}
    If the process $P$ goes $wrong$ with the input $M$ in exchange for $y$, then the input will go $wrong$. An example is that process $P$ expected $y$ to be of type $num$ but $M$ is of type $str$.\\

    \begin{align*}
        \tag{Parallel}\frac{E \vdash P \ra wrong}{E \vdash P \para Q \ra wrong}
    \end{align*}
    If the process $P$ goes $wrong$, then $P$ and $Q$ in parallel will go $wrong$. By applying the rule ParaCommunative we can show that either if $P$ or $Q$ goes $wrong$, $P$ and $Q$ in parallel will go $wrong$.\\

    \begin{align*}
        \tag{Structual Congruence}\frac{E \vdash Q \ra wrong \quad P \equiv Q}{E \vdash P \ra wrong}
    \end{align*}
    If $P$ and $Q$ are structural congruent and $Q$ goes $wrong$, then $P$ also goes $wrong$.\\

    \begin{align*}
        \tag{Drop 1}\frac{\drop{x} \equiv P \quad E \vdash P \ra wrong}{E \vdash \drop{x} \ra wrong}
    \end{align*}
    
    If a dropped name $x$ is structural congruent with process $P$ and $P$ goes $wrong$, then the dropped $x$ will also go $wrong$.\\

    \begin{align*}
        \tag{Drop 2}\frac{\quot{P} \equiv _N x \quad E \vdash P \ra wrong}{E \vdash \drop{x} \ra wrong}
    \end{align*}
    If a quoted process $P$ is name equivalent with a name $x$ and $P$ goes $wrong$. Then the dropped $x$ will also go $wrong$.\\

    \begin{align*}
        \tag{Condition}\frac{E \vdash P \ra wrong}{E \vdash [\phi]P \ra wrong}
    \end{align*}
    If the process $P$ goes $wrong$ then the condition of $P$ will also go $wrong$ even if the condition is false, and thereby never executing $P$.


\subsection{Theorems}
A well-typed process can never break the integrity of a protocol if the following two theorems hold.

\begin{theorem}[Subject Reduction]
	If $E \vdash P$ and $P\ra P'$ then $E \vdash P'$.
\end{theorem}

The first theorem states that if a well-typed process is reduced, the resulting process is well-typed. This ensures that a well-typed process can not reduce to a process that is not well-typed.

\begin{theorem}[Not Wrong]
	If $E \vdash P$ then $E \vdash P \nrightarrow wrong$
\end{theorem}

The second theorem states that a well-typed process can not go $wrong$. Together with the first theorem, this ensures that any process that starts out well-typed can never go $wrong$. Therefore if we prove both theorems, then a well-typed process can never break the integrity of a protocol.
\subsection{Theorem Proofs}
\begin{proof}[Subject Reduction]\\
	We will prove that if $E \vdash P$ and $P\ra P'$ then $E \vdash P'$. This will be proven by induction in the reduction rules of our extended calculus.
	\begin{description}
	\item[Parallel]
		\begin{align*}
		\frac{P \ra P' \quad}{P \para Q \ra P' \para Q} \tag{Parallel}
		\end{align*}\\
		
		Therefore if $E \vdash P \para Q$ then $E \vdash P' \para Q$. The type rule for parallel says $\frac{E \vdash P\quad E \vdash Q}{E \vdash P \para Q}$ so if $E \vdash P'$ then $\frac{E \vdash P' \quad E \vdash Q}{E \vdash P' \para Q}$\\
	\item[Communication]
		\begin{align*}
		\infrule{}{\lift{x}{M}\para\inp{x}{y}P\ra P\{M/y\}} \tag{Communication}
		\end{align*}
		
		\begin{lemma}[Substitution]
			If $E,x:T \vdash P$ and $E_2 \vdash M:T$ then $E, E_2 \vdash P\{M/x \}$
		\end{lemma}
	\item[Structural Congruent]
		\begin{align*}
		\infrule{P \equiv P' \quad P' \ra Q' \quad Q' \equiv Q}{P \ra Q} \tag{Structural Congruent}
		\end{align*}
		
		
		If $E \vdash P$ then $E \vdash P'$ We will want to show that if $E \vdash Q$ then $E \vdash Q'$
		
		\begin{lemma}[Equivalent]
			If $P \equiv Q$ and $E \vdash P$ then $E \vdash Q$
		\end{lemma}
		
		By applying lemma equivalent to the structural congruent, we see that because $P \equiv Q$ and $E \vdash P$ then $E \vdash Q$ and because $P \ra P'$ and $E \vdash P'$ then $E \vdash Q'$.
	\end{description}
	Since the statement, if $E \vdash P$ and $P\ra P'$ then $E \vdash P'$, holds true by induction in the reduction rules of our extended calculus, then it must hold true for all possible reductions, therefore proving the theorem.\qed
\end{proof}
\vspace{10mm}
\begin{proof}[Not Wrong]\\
	We will prove that if $E \vdash P$ then $E \vdash P \nrightarrow wrong$. This will be proven by induction in the process rules of our type system.
	\begin{description}
	\item[Nil]
		Since there is not a wrong-predicate for nil, nil can not go wrong therefore:
		\begin{align*}
		E \vdash 0 \nrightarrow 0
		\end{align*}
	\item[Parallel]
		\begin{align*}
		\frac{E \vdash P \quad E \vdash Q }{E \vdash P \para Q}
		\end{align*}
		
		By induction hypothesis we have that: 
		\begin{align*}
		E \vdash P \nrightarrow wrong\\
		E \vdash Q \nrightarrow wrong
		\end{align*}
		Therefore:
		\begin{align*}
		\frac{}{E \vdash P \para Q \nrightarrow wrong}
		\end{align*}
		
		Since the wrong-predicate for parallel says:
		\begin{align*}
		\frac{E \vdash P \ra wrong}{E \vdash P \para Q \ra wrong}
		\end{align*}
		And we can prove by the \ref{rule:ParaCommutative} rule that the same holds for $Q$.
	\item[Lift]
		\begin{align*}
		\frac{E \vdash M:T_1 \quad E \vdash x:ch[T_2]^l \quad T_1 = T_2}{E \vdash \lift{x}{M}}
		\end{align*}
		By induction hypothesis we have that:
		\begin{align*}
		&E \vdash M:T_1\\
		&E \vdash x:ch[T_2]^l\\
		&T_1 = T_2
		\end{align*}
		Therefore:
		\begin{align*}
		\frac{}{E \vdash \lift{x}{M} \nrightarrow wrong}
		\end{align*}
		
		Since the wrong-predicate for Lift says:
		\begin{align*}
		\frac{E \nvdash M}{E \vdash \lift{x}{M} \ra wrong}
		\end{align*}
		\begin{align*}
		\frac{E \vdash M:T_1 \quad E \vdash x:ch[T_2]^l \quad T_1 \neq T_2}{E \vdash \lift{x}{M} \ra wrong}
		\end{align*}
	\item[Input]
		\begin{align*}
		\frac{E \vdash x:ch[T]^l \quad E,y:T \vdash P}{E \vdash \inp{x}{y}P}
		\end{align*}
		By induction hypothesis we have that:
		\begin{align*}
		&E \vdash x:ch[T]^l\\
		&E \vdash y:T \vdash P \nrightarrow wrong
		\end{align*}
		
		Therefore:
		\begin{align*}
		\frac{}{E \vdash \inp{x}{y}P \nrightarrow wrong}
		\end{align*}
		
		Since the wrong-predicate for input says:
		\begin{align*}
		\frac{E \vdash P\{M/y\}\ra wrong \quad E\vdash M\vdash T \quad E(x)=ch[T]^l}{E \vdash \inp{x}{y}P \ra wrong}
		\end{align*}
	\item[Drop]
		\begin{align*}
		\frac{E \vdash P}{E \vdash \drop{x}}
		\end{align*}
		
		By induction hypothesis we have that:
		\begin{align*}
		E \vdash P \nrightarrow wrong
		\end{align*}
		
		Therefore:
		\begin{align*}
		\frac{}{E \vdash \drop{x} \nrightarrow wrong}
		\end{align*}
		
		Since the wrong-predicate for drop says:
		\begin{align*}
		\frac{\drop{x} \equiv P \quad E \vdash P \ra wrong}{E \vdash \drop{x} \ra wrong}
		\end{align*}
%	\item[Quote]
%		\begin{align*}
%		\frac{E \vdash x:T \quad \quot{P} \equiv _N x}{E \vdash \quot{P}:T}
%		\end{align*}
		
%		By induction hypothesis we have that:
%		\begin{align*}
%		E \vdash P \nrightarrow wrong
%		\end{align*}
		
%		Therefore:
%		\begin{align*}
%		\frac{}{E \vdash \quot{P} \nrightarrow wrong}
%		\end{align*}
		
%		Since the wrong-predicate for quote says:
%		\begin{align*}
%		\frac{\quot{P} \equiv _N x \quad E \vdash P \ra wrong}{E \vdash \drop{x} \ra wrong}
%		\end{align*}
	\item[Condition]
		\begin{align*}
		\frac{E \vdash \phi : bool \quad E \vdash P}{E \vdash [\phi]P}
		\end{align*}
		By induction hypothesis we have that:
		\begin{align*}
		E \vdash P \nrightarrow wrong
		\end{align*}
		
		Therefore:
		\begin{align*}
		\frac{}{E \vdash [\phi ]P \nrightarrow wrong}
		\end{align*}
		
		Since the wrong-predicate for condition is:
		\begin{align*}
		\frac{E \vdash P \ra wrong}{E \vdash [\phi ]P \ra wrong}
		\end{align*}
	\end{description}
	Since the statement, if $E \vdash P$ then $E \vdash P \nrightarrow wrong$, holds true by induction in the process rules of our type system, then it must hold true for all possible processes, therefore proving the theorem.\qed
\end{proof}
