\section{Type Security}

We need to define the wrong predicate, to know when a process can go \ensuremath{wrong}. If a process goes \ensuremath{wrong}, it have consequences for it integrity. When a process goes \ensuremath{wrong} it can also be determined that the process have a runtime error.
\begin{description}
\item[\textbf{Definition: Wrong}]
    \ensuremath{E \vdash P \ra wrong} \\
    
    To prove when \ensuremath{P} goes \ensuremath{wrong}, every \ensuremath{wrong} case will be presented.\\\\
    \item[\textbf{Lift 1}] \begin{align*}
        \frac{E \nvdash M}{E \vdash \lift{x}{M} \ra wrong}
    \end{align*}\\
        
    \item[\textbf{Lift 2}] \begin{align*}
        \frac{E \vdash M:T_1 \quad E(x)=ch[T_2]}{E \vdash \lift{x}{M} \ra wrong} &\quad where T_1 \neq T_2
    \end{align*}

    \item[\textbf{Input}] \begin{align*}
        \frac{\exists M,\quad E \vdash P\{ M/ y\} \ra wrong \quad E \vdash M:T \quad E(x)= ch[T]} {E \vdash \inp{x}{y}P \ra wrong}\\
    \end{align*}

    \item[\textbf{Parallel}] \begin{align*}
        \frac{E \vdash P \ra wrong}{E \vdash P \para Q \ra wrong}
    \end{align*}

    \item[\textbf{Structural Congruens}] \begin{align*}
        \frac{E \vdash Q \quad P \equiv Q}{E \vdash P \ra wrong}
    \end{align*}

    \item[\textbf{Drop}]\begin{align*}
        \frac{\drop{x} \equiv P \quad E \vdash P \ra wrong}{E \vdash \drop{x} \ra wrong}
    \end{align*}

    \item[\textbf{Quote}]\begin{align*}
        \frac{\quot{P} \equiv _N x \quad E \vdash P \ra wrong}{E \vdash \drop{x} \ra wrong}
    \end{align*}

    \item[\textbf{Condition}]\begin{align*}
        \frac{E \vdash P \ra wrong}{E \vdash [\phi]P \ra wrong}
    \end{align*}
\end{description}

\begin{theorem}[Subject Reduction]
    If \ensuremath{E \vdash P} and \ensuremath{P\ra P'} then \ensuremath{E \vdash P'}.
    Proofs for parallel, communication and structural congruent proving that \ensuremath{P \ra P'} holds will be presented.\\

    \begin{proof}[Parallel]
        \begin{align*}
            \frac{P \ra P' \quad}{P \para Q \ra P' \para Q} \tag{Parallel}
        \end{align*}\\

    Therefor if \ensuremath{E \vdash P \para Q} then \ensuremath{E \vdash P' \para Q}. The type rule for parallel says \ensuremath{\frac{E \vdash P\quad E \vdash Q}{E \vdash P \para Q}} so if \ensuremath{E \vdash P'} then \ensuremath{\frac{E \vdash P' \quad E \vdash Q}{E \vdash P' \para Q}}\\
        \qed
    \end{proof}

    \begin{proof}[Communication]
        \begin{align*}
            \infrule{}{\lift{x}{M}\para\inp{x}{y}P\ra P\{M/y\}} \tag{Communication}
        \end{align*}

        \begin{lemma}[Substitution]
            If \ensuremath{E,x:T \vdash P} and \ensuremath{E_2 \vdash M:T} then \ensuremath{E, E_2 \vdash P\{M/x \}}
        \end{lemma}
        \qed
    \end{proof}

    \begin{proof}[Structural Congruent]
        \begin{align*}
            \infrule{P \equiv P' \quad P' \ra Q' \quad Q' \equiv Q}{P \ra Q} \tag{Structural Congruent}
        \end{align*}

        If \ensuremath{E \vdash P} then \ensuremath{E \vdash P'} We will want to show that if \ensuremath{E \vdash Q} then \ensuremath{E \vdash Q'}

        \begin{lemma}[Equivalent]
            If \ensuremath{P \equiv Q} and \ensuremath{E \vdash P} then \ensuremath{E \vdash Q}
        \end{lemma}

        By applying lemma equivalent to the structural congruent, we see that because \ensuremath{P \equiv Q} and \ensuremath{E \vdash P} then \ensuremath{E \vdash Q} and because \ensuremath{P \ra P'} and \ensuremath{E \vdash P'} then \ensuremath{E \vdash Q'}.
        \qed
    \end{proof}
\end{theorem}
\newpage

\begin{theorem}
    If \ensuremath{E \vdash P} then \ensuremath{E \vdash P \nrightarrow wrong}.\\
    %We need to show by induction that every reduction holds by their assumption and do not goes to \ensuremath{wrong}.
    %Proofs by induction will be shown, proving that every reduction holds and therefor can not go \ensuremath{wrong}\\
    Using proof by induction, proofs for every reduction rule proving that they hold and therefor can not go \ensuremath{wrong} will be presented.

    \begin{proof}[Nil]
        Since there is not a wrong predicate for nil, nil can not go wrong therefor \ensuremath{E \vdash 0 \nrightarrow 0}.
        \qed
    \end{proof}

    \begin{proof}[Parallel]
        \begin{align*}
            \frac{E \vdash P \quad E \vdash Q }{E \vdash P \para Q}
        \end{align*}

        By induction hypothesis we have that: 
        \begin{align*}
            E \vdash P \nrightarrow wrong\\
            E \vdash Q \nrightarrow wrong
        \end{align*}
        Therefor:
        \begin{align*}
            \frac{}{E \vdash P \para Q \nrightarrow wrong}
        \end{align*}

        Since the wrong predicate for parallel says:
        \begin{align*}
            \frac{E \vdash P \ra wrong}{E \vdash P \para Q \ra wrong}
        \end{align*}
        And we can prove by the rule ParaCommutative that the same holds for \ensuremath{Q}.\\
        \qed
    \end{proof}

    \begin{proof}[Lift]
        \begin{align*}
            \frac{E \vdash M:T_1 \quad E \vdash x:ch[T_2] \quad T_1 = T_2}{E \vdash \lift{x}{M}}
        \end{align*}
        By induction hypothesis we have that:
        \begin{align*}
            &E \vdash M:T_1\\
            &E \vdash x:ch[T_2]\\
            &T_1 = T_2
        \end{align*}
        Therefor:
        \begin{align*}
            \frac{}{E \vdash \lift{x}{M} \nrightarrow wrong}
        \end{align*}

        Since the wrong predicate for Lift says:
        \begin{align*}
            \frac{E \nvdash M}{E \vdash \lift{x}{M} \ra wrong}
        \end{align*}
        \begin{align*}
            \frac{E \vdash M:T_1 \quad E \vdash x:ch[T_2] \quad T_1 \neq T_2}{E \vdash \lift{x}{M} \ra wrong}
        \end{align*}
        \qed
    \end{proof}

    \begin{proof}[Input]
        \begin{align*}
            \frac{E \vdash x:ch[T] \quad E,y:T \vdash P}{E \vdash \inp{x}{y}P}
        \end{align*}
        By induction hypothesis we have that:
        \begin{align*}
            &E \vdash x:ch[T]\\
            &E \vdash y:T \vdash P \nrightarrow wrong
        \end{align*}

        Therefor:
        \begin{align*}
            \frac{}{E \vdash \inp{x}{y}P \nrightarrow wrong}
        \end{align*}

        Since the wrong predicate for input says:
        \begin{align*}
            \frac{\exists M, \quad E \vdash P\{M/y\}\ra wrong \quad E\vdash M\vdash T \quad E(x)=ch[T]}{E \vdash \inp{x}{y}P \ra wrong}
        \end{align*}
        \qed
    \end{proof}

    \begin{proof}[Drop]
        \begin{align*}
            \frac{E \vdash P}{E \vdash \drop{x}}
        \end{align*}

        By induction hypothesis we have that:
        \begin{align*}
            E \vdash P \nrightarrow wrong
        \end{align*}

        Therefor:
        \begin{align*}
            \frac{}{E \vdash \drop{x} \nrightarrow wrong}
        \end{align*}

        Since the wrong predicate for drop says:
        \begin{align*}
            \frac{\drop{x} \equiv P \quad E \vdash P \ra wrong}{E \vdash \drop{x} \ra wrong}
        \end{align*}
        \qed\\
    \end{proof}

    \begin{proof}[Quote]
        \begin{align*}
            \frac{E \vdash x:T \quad \quot{P} \equiv _N x}{E \vdash \quot{P}:T}
        \end{align*}

        By induction hypothesis we have that:
        \begin{align*}
            E \vdash \drop{x} \nrightarrow wrong
        \end{align*}

        Therefor:
        \begin{align*}
            \frac{}{E \vdash \drop{x} \nrightarrow wrong}
        \end{align*}

        Since the wrong predicate for quote says:
        \begin{align*}
            \frac{\quot{P} \equiv _N x \quad E \vdash P ra wrong}{E \vdash \drop{x} \ra wrong}
        \end{align*}
        \qed
    \end{proof}

    \begin{proof}[Condition]
        \begin{align*}
            \frac{E \vdash \phi : bool \quad E \vdash P}{E \vdash [\phi]P}
        \end{align*}
        By induction hypothesis we have that:
        \begin{align*}
            E \vdash P \nrightarrow wrong
        \end{align*}

        Therefor:
        \begin{align*}
            \frac{}{E \vdash [\phi ]P \nrightarrow wrong}
        \end{align*}

        Since the wrong predicate for condition is:
        \begin{align*}
            \frac{E \vdash P \ra wrong}{E \vdash [\phi ]P \ra wrong}
        \end{align*}
        \qed
    \end{proof}
\end{theorem}




%\begin{align*}
%    \tag{Wrong}& \infrule{E(x) = ch[T_1] \quad E \vdash M:T_2 \quad \level{T_2} > \level{T_1} }{\lift{x}{M} \ra wrong}
%\end{align*}


