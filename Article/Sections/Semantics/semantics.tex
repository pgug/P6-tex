\subsection{Equivalence Relations}
The structural congruence of processes is denoted by the relation $\equiv$.

\begin{figure}[h]
    \begin{align*}
        P\para 0 \equiv\ &P \equiv 0\para P\\
        P\para Q &\equiv Q\para P\\
        P\para (Q\para R)&\equiv (P\para Q)\para R
    \end{align*}
    \caption{The rules of structural congruence for processes}
\end{figure}

%\begin{figure}[h]
%    \begin{center}
%        \begin{tabular}[c]{ll}
%            \runa{Equivalence} & \infrule{P\equiv P'\quad P'\ra Q'\quad Q'\equiv Q}{P\ra Q}
%        \end{tabular}
%    \end{center}
%    \caption{Structural Congruence for Equivalence}
%    \label{fig:equi}
%\end{figure}

\FloatBarrier

The equivalence of names is denoted by the relation $\equiv _N$

\begin{figure}[h]
	\begin{align}
	& \infrule{}{\quot{\drop{x}}\equiv _N x} \tag{Drop and Quote}
	\end{align}
	\caption{Name equivalence rule for drop and quote}
	\label{fig:dropquot}
\end{figure}

\noindent
The rule in \figref{fig:dropquot} states that if $x$ is dropped, and then quoted, then it should stay name equivalent with $x$.

\begin{figure}[h]
	\begin{align}
	& \infrule{P\equiv Q}{\quot{P} \equiv _N \quot{Q}} \tag{Structural equivalence}
	\end{align}
	\caption{Name equivalence rule for structural congruence}
	\label{fig:strucequiv}
\end{figure}

\noindent
The rule in \figref{fig:strucequiv} states, that if $P$ and $Q$ are structural congruent with each other, then the quoted $P$ and $Q$ are name equivalent with each other.

\FloatBarrier

\subsection{Reduction Rules}
We make use of the following reduction rules.

\subsubsection{Base rules}
These are similar to the reduction rules of the rho-calculus as described in \citep{Meredith2005}. The only difference is the communication rule reflecting the syntax changes.

\begin{figure}[h]
\begin{align}
\tag{Comm} \infrule{x_0 \equiv_N x_1}{\lift{x_0}{M}\para\inp{x_1}{y}P\ra P\{M/y\}}&\\
\tag{Parallel} \infrule{P\ra P'}{P\para Q\ra P'\para Q}&\\
\tag{Equivalence} \infrule{P\equiv P'\quad P'\ra Q'\quad Q'\equiv Q}{P\ra Q}&
\end{align}
\end{figure}

%\begin{figure}[h]
%	\begin{align}
%	& \infrule{x_0 \equiv_N x_1}{\lift{x_0}{M}\para\inp{x_1}{y}P\ra P\{M/y\}} \tag{Comm}
%	\end{align}
%	\caption{Communication rule for when a term is lifted instead of a process.}
%	\label{fig:com}
%\end{figure}
%
%
%\begin{figure}[!h]
%	\begin{align}
%	& \infrule{P\ra P'}{P\para Q\ra P'\para Q} \tag{Parallel}
%	\end{align}
%	\caption{Reduction rule for parallel processes}
%	\label{fig:para}
%\end{figure}
%
%\begin{figure}[h]
%	\begin{align}
%		& \infrule{P\equiv P'\quad P'\ra Q'\quad Q'\equiv Q}{P\ra Q} \tag{Equivalence}
%	\end{align}
%	\caption{Reduction rule for structural congruence}
%	\label{fig:equi}
%\end{figure}


\FloatBarrier

\subsubsection{Rules for conditional processes}
The reduction rule of a given conditional process is decided by its boolean expression. These are evaluated using common boolean arithmetics.

\begin{figure}[h]
	\begin{align}
	& \infrule{}{[\top]P\ra P} \tag{True}
	\end{align}
	\caption{Reduction rule for conditional processes given true}
	\label{fig:true}
\end{figure}

\begin{figure}[h]
	\begin{align}
	& \infrule{}{[\bot]P\ra \nil} \tag{False}
	\end{align}
	\caption{Reduction rule for conditional processes given false}
	\label{fig:false}
\end{figure}

\FloatBarrier

\section{Evaluating Terms}
In order to make use of our terms we need to define how to evaluate them.
The operation terms are intended to operate on terms, so it is sufficient to define how to evaluate these operations.
Operations can also define abstract datatypes which we will make use of.
The operations defined in this section are chosen specifically for modeling a blockchain protocol and could be different given another goal.

\subsubsection{Tuple operation}
The operation for getting Terms out of a tuple could look like this recursive operation, where \textit{n} determines the index starting from 1.
\begin{figure}[h]
    \begin{align*}
        &first \quad (M_1, M_2) = M_1\\
        &second \quad (M_1, M_2) = M_2\\
        &f(1) \eqdef first(M_1, M_2)\\
        &f(n) \eqdef second(M_1, f(n-1))\\
        &index( \quad M, n) \eqdef [n = 1]first(M)\para [n > 1]index(second(M), n-1)
    \end{align*}
    \caption{The operation for indexing a tuple. Where \textit{M} is a tuple, and \textit{n} is a number bigger than  or equal to 1}
\end{figure}
\FloatBarrier


\subsection{Encryption and Decryption}
There exist no general way of retrieving the term used in an operation.
This enables us to encrypt a term by using it in an encryption operation.
We also define a decrypt operation that allows retrieval of the encrypted term.
Together with a key we can define operations for symmetric-key encryption with the rule in \figref{decryptrule}.

\begin{figure}[h]
    \begin{align*}
        &\mop{dec}(\mop{enc}(M_1, M_2),M_2) = M_1 \tag{Decrypt}
    \end{align*}
    \caption{Operations for encrypting and decrypting a message $M_1\ \mathrm{with\ key}\ M_2$.}
    \label{decryptrule}
\end{figure}
\FloatBarrier

\subsection{List}
The list operations enables the abstraction of lists.
They are described as follows:

\begin{description}
	\item[\op{head}(list)] The first element of the list
	\item[\op{tail}(list)] The list without its first element
	\item[\op{append}(x, list)] The list with element $x$ appended to its beginning.
\end{description}

These operations are defined by the rules of \figref{listoprules}.

\begin{figure}[h]
	\begin{align*}
		&\mop{head}(\mop{append}(M_1, M_2)) = M_1 \tag{Head} \\
		&\mop{tail}(\mop{append}(M_1, M_2)) = M_2 \tag{Tail}
	\end{align*}
	\caption{The rules of the list operations}
	\label{listoprules}
\end{figure}
\FloatBarrier

\section{Blockchain}

The blockchain operations enables the abstraction of blockchains.
They are described as follows:

\begin{description}
	\item[\op{newBlock}(data, previous)]
	A block that succeeds previous and contains data.
	\item[\op{getLatest}(chain)]
	The last block of the chain
	\item[\op{addToChain}(block, chain)]
	The chain with block added to it.
	\item[\op{getLength}(chain)]
	The length of the chain represented as a number.
	\item[\op{isValid}(chain)]
	The validity of the chain represented as a boolean.
\end{description}

