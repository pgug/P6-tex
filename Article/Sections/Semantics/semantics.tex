\subsection{Equivalence Relations}
The structural congruence of processes, denoted by the relation $\equiv$, allows us to express if two processes have the same structure. It is defined by the following rules.

\begin{align*}
    \tag{ParaIdentity}P&\equiv P\para 0\\
    \tag{ParaCommutative}\label{rule:ParaCommutative}P\para Q &\equiv Q\para P\\
    \tag{ParaAssociative}P\para (Q\para R)&\equiv (P\para Q)\para R
\end{align*}


\FloatBarrier

The equivalence of names, denoted by the relation $\equiv _N$, allows us to express if two names are the same. It is defined by the following rules.

\begin{align}
    \tag{DropAndQuote}\label{rule:DropAndQuote}\infrule{}{\quot{\drop{x}}\equiv _N x}
\end{align}

\noindent
The \ref{rule:DropAndQuote} rule states, that if $x$ is dropped, and then quoted, then it should stay name equivalent with $x$.

\begin{align}
    \tag{StructuralEquivalence}\label{rule:StructuralEquivalence}\infrule{P\equiv Q}{\quot{P} \equiv _N \quot{Q}}
\end{align}

\noindent
The \ref{rule:StructuralEquivalence} rule states, that if $P$ and $Q$ are structural congruent with each other, then the quoted $P$ and $Q$ are name equivalent with each other.

\FloatBarrier

\subsection{Name substitution} The name substitution $P\{x/y\}$ denotes that every occurrence of the name $y$ in the process $P$ is substituted by the name $x$.

\subsection{Reduction Rules}
We make use of the following reduction rules.

\subsubsection{Base rules}
These are similar to the reduction rules of the rho-calculus as described in \citep{Meredith2005}. The only difference is the communication rule reflecting the syntax changes.

\begin{align}
    \tag{Comm} \infrule{x_0 \equiv_N x_1}{\lift{x_0}{M}\para\inp{x_1}{y}P\ra P\{M/y\}}&\\
    \tag{Parallel} \infrule{P\ra P'}{P\para Q\ra P'\para Q}&\\
    \tag{Equivalence} \infrule{P\equiv P'\quad P'\ra Q'\quad Q'\equiv Q}{P\ra Q}&
\end{align}

\FloatBarrier

\subsubsection{Rules for conditional processes}
The reduction rule of a given conditional process is decided by its boolean expression. These are evaluated using common boolean arithmetics.

\begin{align}
    & \infrule{\phi \ra \top}{[\phi]P\ra P} \tag{True}\\
    & \infrule{\phi \ra \bot}{[\phi]P\ra \nil} \tag{False}
\end{align}

The rules for boolean logic, "and", "or", and "negation", is shown here, under what condition it is evaluated to true (\ensuremath{\top}) and false (\ensuremath{\bot}).
\begin{align*}
    \infrule{\phi_1 \ra \top \quad \phi_2 \ra \top}{\phi_1 \wedge \phi_2 \ra \top}\tag{And True}\\
    \infrule{\phi_i \ra \bot}{\phi_1 \wedge \phi_2 \ra \bot} &\text{where}\ i\in \{1,2\}\tag{And False}\\
    \infrule{\phi_i \ra \top}{\phi_1 \vee \phi_2 \ra \top} &\text{where}\ i\in \{1,2\} \tag{Or True}\\
    \infrule{\phi_1 \ra \bot \quad \phi_2 \ra \bot}{\phi_1 \vee \phi_2 \ra \bot} \tag{Or False}\\
    \infrule{\phi \ra \bot}{\neg\phi \ra \top} \tag{Negate True}\\
    \infrule{\phi \ra \top}{\neg\phi \ra \bot} \tag{Negate False}\\
\end{align*}

The rules for arithmetic logic, equal to, greater than, etc., does not specify what happens if the terms are not comparable, like numbers and strings.


\begin{align*}
    \infrule{}{M_1 = M_2 \ra \top}& \text{if }M_1=M_2 \tag{Equal True}\\
    \infrule{}{M_1 = M_2 \ra \bot}& \text{if }M_1\neq M_2 \tag{Equal False}\\
    \infrule{}{M_1 \neq M_2 \ra \top}& \text{if }M_1\neq M_2 \tag{NotEqual True}\\
    \infrule{}{M_1 \neq M_2 \ra \bot}& \text{if }M_1= M_2 \tag{NotEqual False}\\
    \infrule{}{M_1 < M_2 \ra \top}& \text{if }M_1<M_2 \tag{Less True}\\
    \infrule{}{M_1 < M_2 \ra \bot}& \text{if }M_1\nless M_2 \tag{Less False}\\
    \infrule{}{M_1 > M_2 \ra \top}& \text{if }M_1>M_2 \tag{Greater True}\\
    \infrule{}{M_1 > M_2 \ra \bot}& \text{if }M_1\ngtr M_2 \tag{Greater False}\\
    \infrule{}{M_1 \leq M_2 \ra \top}& \text{if }M_1\leq M_2 \tag{Less Equal True}\\
    \infrule{}{M_1 \leq M_2 \ra \bot}& \text{if }M_1\nleq M_2 \tag{Less Equal False}\\
    \infrule{}{M_1 \geq M_2 \ra \top}& \text{if }M_1\geq M_2 \tag{Greater Equal True}\\
    \infrule{}{M_1 \geq M_2 \ra \bot}& \text{if }M_1\ngeq M_2 \tag{Greater Equal False}
\end{align*}


\FloatBarrier
\subsection{Replication}
Our implementation makes use of replication, denoted by $!P$, to express inexhaustible processes.
The replication $!P$ proceeds as an arbitrary number of copies of $P$ in parallel. This is shown by the reduction example in \figref{fig:reductionexample}. The replication was developed by L. G. Meredith and M. Radestock\citep{Meredith2005}.
%\begin{figure}[h]
%    \begin{center}
%        \begin{tabular}[c]{cll}
%            & !P & \runa{Initial} \\

%            $\equiv$ & \lift{x}{\inp{x}{y}(\lift{x}{\drop{y}}\para\drop{y})\para P}\para\inp{x}{y}(\lift{x}{\drop{y}}\para \drop{y}) & \runa{Substitution} \\

%            $\ra$ & \lift{x}{\drop{\quot{\inp{x}{y}(\lift{x}{\drop{y}}\para\drop{y})\para P}}}\para\drop{\quot{\inp{x}{y}(\lift{x}{\drop{y}}\para\drop{y})\para P}} & \runa{Communication} \\

%            $\equiv$ & \lift{x}{\inp{x}{y}(\lift{x}{\drop{y}}\para\drop{y})\para P}\para\inp{x}{y}(\lift{x}{\drop{y}}\para \drop{y})\para P & \runa{DropQuote} \\

%            $\equiv$ & !P\para P & \runa{Substitution}
%        \end{tabular}
%    \end{center}
%    \caption{Reduction example of replication}
%    \label{fig:reductionexample}
%\end{figure}


\begin{figure}[h]
    \begin{align}
        &!P \tag{Initial} \\
        &\equiv \lift{x}{\inp{x}{y}(\lift{x}{\drop{y}}\para\drop{y})\para P}\para\inp{x}{y}(\lift{x}{\drop{y}}\para \drop{y}) \tag{Substitution} \\
        &\ra \lift{x}{\drop{\quot{\inp{x}{y}(\lift{x}{\drop{y}}\para\drop{y})\para P}}}\para\drop{\quot{\inp{x}{y}(\lift{x}{\drop{y}}\para\drop{y})\para P}} \tag{Comm}\\
        &\equiv \lift{x}{\inp{x}{y}(\lift{x}{\drop{y}}\para\drop{y})\para P}\para\inp{x}{y}(\lift{x}{\drop{y}}\para \drop{y})\para P \tag{DropQuote} \\
        &\equiv\ 
        !P\para P \tag{Substitution}
    \end{align}
    \caption{Reduction example of replication}
    \label{fig:reductionexample}
\end{figure}

\FloatBarrier

\section{Evaluating Terms}
In order to make use of our terms we need to define how to evaluate them.
The operation terms are intended to operate on terms, so it is sufficient to define how to evaluate these operations.
Operations can also define abstract datatypes which we will make use of.
The operations defined in this section are chosen specifically for modeling a blockchain protocol and could be different given another goal.

\subsubsection{Tuple operation}
The operation for getting Terms out of a tuple could look like this recursive operation, where \textit{n} determines the index starting from 1.
\begin{figure}[h]
    \begin{align*}
        &first \quad (M_1, M_2) = M_1\\
        &second \quad (M_1, M_2) = M_2\\
        &f(1) \eqdef first(M_1, M_2)\\
        &f(n) \eqdef second(M_1, f(n-1))\\
        &index( \quad M, n) \eqdef [n = 1]first(M)\para [n > 1]index(second(M), n-1)
    \end{align*}
    \caption{The operation for indexing a tuple. Where \textit{M} is a tuple, and \textit{n} is a number bigger than  or equal to 1}
\end{figure}
\FloatBarrier


\subsection{Encryption and Decryption}
There exist no general way of retrieving the term used in an operation.
This enables us to encrypt a term by using it in an encryption operation.
We also define a decrypt operation that allows retrieval of the encrypted term.
Together with a key we can define operations for symmetric-key encryption with the rule in \figref{decryptrule}.

\begin{figure}[h]
    \begin{align*}
        &\mop{dec}(\mop{enc}(M_1, M_2),M_2) = M_1 \tag{Decrypt}
    \end{align*}
    \caption{Operations for encrypting and decrypting a message $M_1\ \mathrm{with\ key}\ M_2$.}
    \label{decryptrule}
\end{figure}
\FloatBarrier

\subsection{List}
The list operations enables the abstraction of lists.
They are described as follows:

\begin{description}
	\item[\op{head}(list)] The first element of the list
	\item[\op{tail}(list)] The list without its first element
	\item[\op{append}(x, list)] The list with element $x$ appended to its beginning.
\end{description}

These operations are defined by the rules of \figref{listoprules}.

\begin{figure}[h]
	\begin{align*}
		&\mop{head}(\mop{append}(M_1, M_2)) = M_1 \tag{Head} \\
		&\mop{tail}(\mop{append}(M_1, M_2)) = M_2 \tag{Tail}
	\end{align*}
	\caption{The rules of the list operations}
	\label{listoprules}
\end{figure}
\FloatBarrier

\section{Blockchain}

The blockchain operations enables the abstraction of blockchains.
They are described as follows:

\begin{description}
	\item[\op{newBlock}(data, previous)]
	A block that succeeds previous and contains data.
	\item[\op{getLatest}(chain)]
	The last block of the chain
	\item[\op{addToChain}(block, chain)]
	The chain with block added to it.
	\item[\op{getLength}(chain)]
	The length of the chain represented as a number.
	\item[\op{isValid}(chain)]
	The validity of the chain represented as a boolean.
\end{description}

