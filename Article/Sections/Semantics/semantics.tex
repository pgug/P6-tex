\subsection{Equivalence Relations}
The structural congruence of processes is denoted by the relation $\equiv$.

\begin{align*}
	P\para 0 \equiv\ &P \equiv 0\para P\\
    P\para Q &\equiv Q\para P\\
    P\para (Q\para R)&\equiv (P\para Q)\para R
\end{align*}


\FloatBarrier

The equivalence of names is denoted by the relation $\equiv _N$

\begin{align}
	& \infrule{}{\quot{\drop{x}}\equiv _N x} \tag{Drop and Quote}
\end{align}

\noindent
The rule states that if $x$ is dropped, and then quoted, then it should stay name equivalent with $x$.

\begin{align}
	& \infrule{P\equiv Q}{\quot{P} \equiv _N \quot{Q}} \tag{Structural equivalence}
\end{align}

\noindent
The rule states, that if $P$ and $Q$ are structural congruent with each other, then the quoted $P$ and $Q$ are name equivalent with each other.

\FloatBarrier

\subsection{Reduction Rules}
We make use of the following reduction rules.

\subsubsection{Base rules}
These are similar to the reduction rules of the rho-calculus as described in \citep{Meredith2005}. The only difference is the communication rule reflecting the syntax changes.

\begin{align}
	\tag{Comm} \infrule{x_0 \equiv_N x_1}{\lift{x_0}{M}\para\inp{x_1}{y}P\ra P\{M/y\}}&\\
	\tag{Parallel} \infrule{P\ra P'}{P\para Q\ra P'\para Q}&\\
	\tag{Equivalence} \infrule{P\equiv P'\quad P'\ra Q'\quad Q'\equiv Q}{P\ra Q}&
\end{align}

\FloatBarrier

\subsubsection{Rules for conditional processes}
The reduction rule of a given conditional process is decided by its boolean expression. These are evaluated using common boolean arithmetics.

\begin{align}
	& \infrule{\phi \ra \top}{[\phi]P\ra P} \tag{True}\\
	& \infrule{\phi \ra \bot}{[\phi]P\ra \nil} \tag{False}
\end{align}
	

\FloatBarrier
We also make use of replication, denoted by $!P$, to express inexhaustible processes.
The replication $!P$ proceeds as an arbitrary number of copies of $P$ in parallel. This is shown by the reduction example in \figref{fig:reductionexample}. The replication was developed by L. G. Meredith and M. Radestock\citep{Meredith2005}.
%\begin{figure}[h]
%    \begin{center}
%        \begin{tabular}[c]{cll}
%            & !P & \runa{Initial} \\
%            $\equiv$ & \lift{x}{\inp{x}{y}(\lift{x}{\drop{y}}\para\drop{y})\para P}\para\inp{x}{y}(\lift{x}{\drop{y}}\para \drop{y}) & \runa{Substitution} \\

%            $\ra$ & \lift{x}{\drop{\quot{\inp{x}{y}(\lift{x}{\drop{y}}\para\drop{y})\para P}}}\para\drop{\quot{\inp{x}{y}(\lift{x}{\drop{y}}\para\drop{y})\para P}} & \runa{Communication} \\

%            $\equiv$ & \lift{x}{\inp{x}{y}(\lift{x}{\drop{y}}\para\drop{y})\para P}\para\inp{x}{y}(\lift{x}{\drop{y}}\para \drop{y})\para P & \runa{DropQuote} \\

%            $\equiv$ & !P\para P & \runa{Substitution}
%        \end{tabular}
%    \end{center}
%    \caption{Reduction example of replication}
%    \label{fig:reductionexample}
%\end{figure}


\begin{figure}[h]
    \begin{align}
        &!P \tag{Initial} \\
        &\equiv \lift{x}{\inp{x}{y}(\lift{x}{\drop{y}}\para\drop{y})\para P}\para\inp{x}{y}(\lift{x}{\drop{y}}\para \drop{y}) \tag{Substitution} \\
        &\ra \lift{x}{\drop{\quot{\inp{x}{y}(\lift{x}{\drop{y}}\para\drop{y})\para P}}}\para\drop{\quot{\inp{x}{y}(\lift{x}{\drop{y}}\para\drop{y})\para P}} \tag{Comm}\\
        &\equiv \lift{x}{\inp{x}{y}(\lift{x}{\drop{y}}\para\drop{y})\para P}\para\inp{x}{y}(\lift{x}{\drop{y}}\para \drop{y})\para P \tag{DropQuote} \\
        &\equiv\ 
        !P\para P \tag{Substitution}
    \end{align}
    \caption{Reduction example of replication}
    \label{fig:reductionexample}
\end{figure}

\FloatBarrier

%\begin{itemize}
%    \item encrypt --- enc
%    \item decrypt --- dec
%    \item equal   --- eq
%    \item !equal  --- !eq
%    \item greater --- >
%    \item greatereq --- >=
%    \item lesser  --- <
%    \item lessereq --- <=
%    \item hashing --- hash
%\end{itemize}

\begin{align}
    &enc(M_1, M_2) = M_2 \tag{Encrypt}\\
    &dec(enc(M_1, M_2),M_2) = M_1 \tag{Decrypt}\\
    &hash(M_1) = M_2 \tag{Hashing}\\
    &[x_1 = x_2]P \ra P \tag{Equal}\\
    &[x_1 \neq x_2]P \ra P \tag{Not Equal}\\
    &[x_1 > x_2]P \ra P \tag{Greater}\\
    &[x_1 >= x_2]P \ra P \tag{Greater than}\\
    &[x_1 < x_2]P \ra P \tag{Lesser}\\
    &[x_1 <= x_2]P \ra P \tag{Lesser than}
\end{align}

