\subsubsection{Projection}
A projection operation for retrieving terms from a tuple is useful when working with tuples. This operation is shown in \figref{tupleop}, where \textit{k} determines the index of the retrieved term.
\begin{figure}[h]
    \begin{align*}
        \pi^k(M)=\pi^k((M_1,..., M_n))=M_k \text{ where } 1\leq k \leq n
    \end{align*}
    \caption{The operation for projecting a tuple, where $M$ is a tuple, and $k$ is a number.}
\label{tupleop}
\end{figure}

In our implementation we use tuples to represent a group of attributes. To project an attribute from a tuple, we use the name of the attribute instead of a number. As an example, projecting the hash of a block can be expressed as $\pi^{hash}(block)$.
\FloatBarrier