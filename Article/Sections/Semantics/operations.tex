\subsection{Evaluating Terms}
In order to make use of our terms we need to define how to evaluate them.
The operation terms are intended to operate on terms, so it is sufficient to define how to evaluate these operations.
Operations can also define abstract datatypes which we will make use of.
The operations defined in this section are chosen specifically for modeling a blockchain protocol and could be different given another goal.

\subsubsection{Tuple operation}
The operation for getting Terms out of a tuple could look like this recursive operation, where \textit{n} determines the index starting from 1.
\begin{figure}[h]
    \begin{align*}
        &first \quad (M_1, M_2) = M_1\\
        &second \quad (M_1, M_2) = M_2\\
        &f(1) \eqdef first(M_1, M_2)\\
        &f(n) \eqdef second(M_1, f(n-1))\\
        &index( \quad M, n) \eqdef [n = 1]first(M)\para [n > 1]index(second(M), n-1)
    \end{align*}
    \caption{The operation for indexing a tuple. Where \textit{M} is a tuple, and \textit{n} is a number bigger than  or equal to 1}
\end{figure}
\FloatBarrier


\subsubsection{Encryption and Decryption}
There exist no general way of retrieving the term used in an operation.
This enables us to encrypt a term by using it in an encryption operation.
We also define a decrypt operation that allows retrieval of the encrypted term.
Together with a key we can define operations for symmetric-key encryption with the rule in \figref{decryptrule}.

\begin{figure}[h]
    \begin{align*}
        &dec(enc(M_1, M_2),M_2) = M_1 \tag{Decrypt}
    \end{align*}
    \caption{Operations for encrypting and decrypting a message $M_1\ \mathrm{with\ key}\ M_2$.}
    \label{decryptrule}
\end{figure}
\FloatBarrier

\subsubsection{List}
The list operations enables the abstraction of lists.
They are described as follows:

\begin{description}
	\item[head(list)] The first element of the list
	\item[tail(list)] The list without its first element
	\item[append(x, list)] The list with element $x$ appended to its beginning.
\end{description}

These operations are defined by the rules of \figref{listoprules}.

\begin{figure}[h]
	\begin{align*}
		&head(append(M_1, M_2)) = M_1 \tag{Head} \\
		&tail(append(M_1, M_2)) = M_2 \tag{Tail}
	\end{align*}
	\caption{The rules of the list operations}
	\label{listoprules}
\end{figure}
\FloatBarrier

\subsection{Blockchain}

The blockchain operations enables the abstraction of blockchains. These operations are also capable of treating a singular block as a blockchain.

\begin{description}
	\item[newBlock(data, previous)]
	A block that succeeds previous and contains data.
	\item[getLatest(chain)]
	The last block of the chain
	\item[addToChain(block, chain)]
	The chain with block added to it.
	\item[getLength(chain)]
	The length of the chain represented as a number.
	\item[isValid(chain)]
	The validity of the chain represented as a boolean.
\end{description}


