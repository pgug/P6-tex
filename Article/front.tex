\mainmatter  % start of an individual contribution

% first the title is needed
\title{Protocol integrity with type-system in Rho-calculus}


\titlerunning{Integrity in Rho-Calculus}


\author{Group: D608F17\\Charlie Dittfeld Byrdam\\Thomas Frandsen\\Jesper Windelborg Nielsen\\Morten Meyer Rasmussen\\Supervisor Hans Hüttel}
\authorrunning{Integrity in Rho-Calculus}

\institute{Aalborg University\\Department of Computer Science\\Selma Lagerløf vej 300\\9220 Aalborg Ø}



\toctitle{}
\tocauthor{}
\maketitle
\begin{abstract}
We approach the problem if an blockchain protocol is a secure technology by implementing a simple blockchain protocol in the rho-calculus. We chose to extend the rho-calculus to make our blockchain implementation easier to design and read, as it is possible to make a implementation without the extensions but that would be far more complex and as the implementation is not the focus of this article, the extensions were added. This articles focuses on checking for the integrity of the blockchain, so to check that we propose a types system. The type system consist of several types we have defined and security levels. The security levels are used to make sure information, that must not be leaked, is only send on a channel if its security level is at least as high.\\
    The type systems validity is proved in the article with proofs by induction, where all cases that a process can go \ensuremath{wrong}, which means that there will not happen a runtime error, can happen. There after proving that our type system can not go \ensuremath{wrong} we have proved that our type system does verify that the integrity of our blockchain is kept. The type system is not limited for usage on a blockchain implementation, it can be used on any implementation written in the rho-calculus.\\
We have achieved proving and designing a type system, that tests for integrity problems for an implementation made in the rho-calculus.
\end{abstract}
\clearpage
